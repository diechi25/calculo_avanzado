\documentclass[11pt]{article}

\usepackage{exsheets}
\usepackage[paper=a4paper, headheight=110pt,showframe=false, 
            layoutvoffset=2em,
            bottom=2cm, top=3.5cm]{geometry}

\usepackage[spanish]{babel}
\usepackage[babel]{microtype}
\usepackage{lipsum}
\usepackage{unicode-math}
% Fonts can be customized here.
\defaultfontfeatures{Mapping=tex-text}
\setmainfont [Ligatures={Common}]{Linux Libertine O}
\setmonofont[Scale=0.9]{Linux Libertine Mono O}
%\usepackage[svgnames]{xcolor} % Gestión de colores
\usepackage{hyperref}
\hypersetup{
  colorlinks=true, linktocpage=true, pdfstartpage=3, pdfstartview=FitV,%
  breaklinks=true, pageanchor=true,%
  pdfpagemode=UseNone, %
  plainpages=false, bookmarksnumbered, bookmarksopen=true, bookmarksopenlevel=1,%
  hypertexnames=true, pdfhighlight=/O,%nesting=true,%frenchlinks,%
  urlcolor=Maroon, linkcolor=RoyalBlue, citecolor=Blue, %pagecolor=RoyalBlue,%
  pdftitle={},%
  pdfauthor={\textcopyright\ C. Manuel Carlevaro},%
  pdfsubject={},%
  pdfkeywords={},%
  pdfcreator={XeLaTeX},%
  pdfproducer={XeLaTeX}%
}

\usepackage{hyperref}
\usepackage{fontawesome, gensymb}
\usepackage{graphicx}
\setlength{\parindent}{3em}
\setlength{\parskip}{1em} 
\usepackage[shortlabels]{enumitem}

%\NewDocumentCommand{\evalat}{sO{\big}mm}{%
  %\IfBooleanTF{#1}
   %{\mleft. #3 \mright|_{#4}}
   %{#3#2|_{#4}}%
%}


\title{Cálculo avanzado}
\author{Dpto. de Ingenería Mecánica}
\date{Clase 6: Errores}


\begin{document}
% \maketitle

\begin{center}
\framebox[1.0\textwidth][c]{
\huge{\textsc{Cálculo Avanzado}} 
}
\end{center} 

\begin{center}
\vspace{\baselineskip}
\Large{\textsc{Departamento de Ingenería Mecánica}} \\
\textsc{Facultad Regional La Plata} \\
\textsc{Universidad Tecnológica Nacional}
\end{center}

% \vspace{1em}

\begin{center}
\begin{tabular}{r l}
    \textbf{Práctica:} & 6 \\
 \textbf{Tema:} & Errores. \\
 \textbf{Profesor Titular:} & Manuel Carlevaro \\
 \textbf{Jefe de Trabajos Prácticos:} & Diego Amiconi \\
 \textbf{Ayudante de Primera:} & Lucas Basiuk 
\end{tabular}\end{center}

\vspace{1em}

\begin{question} % Kreyszig Problem set 19.1 - 1 pg 796
    Escriba los números $84.175$, $-528.685$, $0.000924138$ y $-362005$ como número con fomato de punto flotante, redondeados a cinco cifras significativas.
\end{question}

\begin{question}  % Kreyszig Problem set 19.1 - 9 pg 797 (sol pg 119)
La solución de la ecuación de segundo grado
\[ a x^2 + b x + c = 0 \]
es
\[ x_{1,2} = \frac{1}{2a} \left( -b \pm \sqrt{b^2 - 4 a c}) \right) \]
Alternativamente, dado que $x_1 x_2 = c/a$, si primero obtenemos $x_2$ con la fórmula anterior podemos calcular $x_1$ usando
\[ x_1 = \frac{c}{a x_2} \]

Resuelva $x^2 - 30 x + 1 = 0$ (a) primero con cuatro cifras significativas y (b) con dos cifras significativas.
\end{question}

\begin{question} % Chapra 3.1 pg 99
Convierta los siguientes números binarios a decimales:
\begin{enumerate}[a)]
    \item \texttt{1011001}$_2$
    \item \texttt{110.0101}$_2$
    \item \texttt{0.01011}$_2$
\end{enumerate}
\end{question}

\begin{question} % Chapra 3.1 pg 99
    Los números hexadecimales, o de base 16, son números basados en los dígitos 0, 1, 2, 3, 4, 5, 6, 7, 8, 9, a, b, c, d, e, f. Convierta el número hexadecimal \texttt{2c.0b7}$_{\text{h}}$ a decimal.
\end{question}

\begin{question}
Convertir los siguientes números binarios en representación de punto flotante precisión simple a decimal: 
\begin{enumerate}[a)]
    \item \texttt{01001010010000100011011001000000} % 3181968
    \item \texttt{11000100111100001110000000000000} % -1927
    \item \texttt{01000000010010010000111111011010} % 3.1415926
\end{enumerate}
\end{question}

\begin{question}
Estime el resultado de las siguientes operaciones, con sus correspondientes cotas de error.
\begin{enumerate}[a)]
    \item $3.5 \pm 0.1) + (8.0 \pm 0.2) - (5.0 \pm 0.4)$
    \item $(3.5 \pm 0.1) \times (8.0 \pm 0.2) $
    \item $(3.5 \pm 0.1) \times (8.0 \pm 0.2) / (5.0 \pm 0.4)$
\end{enumerate}
\end{question}

\begin{question} % Taylor 3.31 pg 98 sol pg 317
Un ángulo $\theta$ se mide como $(125 \pm 2)$\textdegree, y su valor se utiliza para calcular $\sen \theta$. Calcule este valor y su incerteza.
\end{question}

\end{document}
