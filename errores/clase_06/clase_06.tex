\documentclass[9pt, aspectratio=169]{beamer}

\usetheme{metropolis}
\setbeamertemplate{itemize items}{\faAngleRight}

\metroset{titleformat=smallcaps,block=fill,numbering=counter,progressbar=frametitle,sectionpage=none}
\setbeamersize{text margin left=5mm,text margin right=5mm} 
% %%%%%%%%%%%%%%%%%%%%%%%%%%%%%%%%%%%%%%%%%%%%%%%%%%%%%%%%%%%%%%%%%%%%%%%%%%%%%%
% \embedvideo{<poster or text>}{<video file (MP4+H264)>}
% \embedvideo*{...}{...}                     % auto-play
%%%%%%%%%%%%%%%%%%%%%%%%%%%%%%%%%%%%%%%%%%%%%%%%%%%%%%%%%%%%%%%%%%%%%%%%%%%%%%

\usepackage[bigfiles]{pdfbase}
\ExplSyntaxOn
\NewDocumentCommand\embedvideo{smm}{
  \group_begin:
  \leavevmode
  \tl_if_exist:cTF{file_\file_mdfive_hash:n{#3}}{
    \tl_set_eq:Nc\video{file_\file_mdfive_hash:n{#3}}
  }{
    \IfFileExists{#3}{}{\GenericError{}{File~`#3'~not~found}{}{}}
    \pbs_pdfobj:nnn{}{fstream}{{}{#3}}
    \pbs_pdfobj:nnn{}{dict}{
      /Type/Filespec/F~(#3)/UF~(#3)
      /EF~<</F~\pbs_pdflastobj:>>
    }
    \tl_set:Nx\video{\pbs_pdflastobj:}
    \tl_gset_eq:cN{file_\file_mdfive_hash:n{#3}}\video
  }
  %
  \pbs_pdfobj:nnn{}{dict}{
    /Type/RichMediaInstance/Subtype/Video
    /Asset~\video
    /Params~<</FlashVars (
      source=#3&
      skin=SkinOverAllNoFullNoCaption.swf&
      skinAutoHide=true&
      skinBackgroundColor=0x5F5F5F&
      skinBackgroundAlpha=0
    )>>
  }
  %
  \pbs_pdfobj:nnn{}{dict}{
    /Type/RichMediaConfiguration/Subtype/Video
    /Instances~[\pbs_pdflastobj:]
  }
  %
  \pbs_pdfobj:nnn{}{dict}{
    /Type/RichMediaContent
    /Assets~<<
      /Names~[(#3)~\video]
    >>
    /Configurations~[\pbs_pdflastobj:]
  }
  \tl_set:Nx\rmcontent{\pbs_pdflastobj:}
  %
  \pbs_pdfobj:nnn{}{dict}{
    /Activation~<<
      /Condition/\IfBooleanTF{#1}{PV}{XA}
      /Presentation~<</Style/Embedded>>
    >>
    /Deactivation~<</Condition/PI>>
  }
  %
  \hbox_set:Nn\l_tmpa_box{#2}
  \tl_set:Nx\l_box_wd_tl{\dim_use:N\box_wd:N\l_tmpa_box}
  \tl_set:Nx\l_box_ht_tl{\dim_use:N\box_ht:N\l_tmpa_box}
  \tl_set:Nx\l_box_dp_tl{\dim_use:N\box_dp:N\l_tmpa_box}
  \pbs_pdfxform:nnnnn{1}{1}{}{}{\l_tmpa_box}
  %
  \pbs_pdfannot:nnnn{\l_box_wd_tl}{\l_box_ht_tl}{\l_box_dp_tl}{
    /Subtype/RichMedia
    /BS~<</W~0/S/S>>
    /Contents~(embedded~video~file:#3)
    /NM~(rma:#3)
    /AP~<</N~\pbs_pdflastxform:>>
    /RichMediaSettings~\pbs_pdflastobj:
    /RichMediaContent~\rmcontent
  }
  \phantom{#2}
  \group_end:
}
\ExplSyntaxOff
%%%%%%%%%%%%%%%%%%%%%%%%%%%%%%%%%%%%%%%%%%%%%%%%%%%%%%%%%%%%%%%%%%%%%%%%%%%%%%

\usepackage{fontspec,minted}
\usepackage[scale=1]{ccicons}
\usepackage{metalogo}
\usepackage{xcolor,colortbl}
\usepackage{multicol,multirow,booktabs}
\usepackage{appendixnumberbeamer}
\usepackage{graphicx}
\usepackage{bm}
\usepackage{fontawesome}
\usepackage{csquotes}
\usepackage[backend=biber, natbib, sorting=nyt, doi=true, url=false, url=false, isbn=false, maxbibnames=10]{biblatex}
\addbibresource{../../utils/refs.bib}

\usepackage[spanish]{babel}
\usepackage{mathtools}
\usefonttheme{professionalfonts}
\usepackage{textcomp}

\setsansfont[BoldFont={Iwona Bold}, Numbers={Lining, Proportional}]{Iwona Light}
% \setmathsfont(Digits)[Numbers={Lining, Proportional}]{Fira Sans Light}
\setmonofont[Scale=MatchLowercase]{DejaVu Sans Mono}

\setbeamercolor{alerted text}{fg=red,bg=black!2}
\setbeamercolor{progress bar}{fg=red,bg=red!2}
\setbeamertemplate{itemize item}{\faCaretRight}
\setbeamertemplate{itemize subitem}{ \faAngleRight}
\setbeamertemplate{blocks}[shadow=false]
\setbeamercolor{block title}{bg=black!30,fg=red}
\setbeamercolor{block body}{bg=black!20,fg=black}
 
\usepackage{gensymb,amssymb}
\usepackage{upquote}
\usepackage{algpseudocode}
\algrenewcommand\algorithmicrequire{\textbf{Requiere}}
\algrenewcommand\algorithmicensure{\textbf{Devuelve}}
%\setbeamertemplate{blocks}[rounded][shadow=false]
\setbeamertemplate{blocks}[shadow=false]

\newcommand{\cx}{\column{0.5\textwidth}}
\newcommand{\cw}[1]{\column{#1\textwidth}}

\author{Manuel Carlevaro}
\date{{\tiny Departamento de Ingeniería Mecánica \\[-1em]
             Grupo de Materiales Granulares - UTN FRLP \\
        \faEnvelope{} manuel.carlevaro@gmail.com \- $\cdot$ \- \faTwitter{} @mcarlevaro}}
\institute{
  \vspace{6em}
  \centering
  {\tiny
  Cálculo Avanzado \enspace • \enspace 2022 \\
    \faLinux \- $\cdot$ \- \fontspec{TeX Gyre Pagella}\XeLaTeX \- $\cdot$ \- \ccbysa }
}

%% Operadores
\DeclareMathOperator{\sen}{sen}
\DeclareMathOperator{\sign}{sign}
\newcommand{\T}[1]{\underline{\bm{#1}}}
\DeclareMathOperator{\Tr}{Tr}

\usepackage{hyperref}
\hypersetup{
    colorlinks,
    citecolor=blue,
    filecolor=black,
    linkcolor=blue,
    urlcolor=blue
}
\urlstyle{same}

%% Códigos
\usepackage{minted}
\newminted[cpp]{cpp}{linenos,fontsize=\footnotesize,frame=lines,numbersep=4pt}
\newmintedfile[cppcode]{cpp}{linenos,fontsize=\footnotesize,frame=lines,numbersep=4pt}
\newcommand{\mic}[1]{\mintinline{C++}{#1}}

\newminted[py]{python}{linenos,fontsize=\footnotesize,frame=lines,numbersep=4pt}
\newminted[pyc]{pycon}{linenos,fontsize=\footnotesize,frame=lines,numbersep=4pt} % Consola de Python
\newminted[ipy3]{ipython3}{linenos,fontsize=\footnotesize,frame=lines,numbersep=4pt} % Consola de iPython3
\newmintedfile[pycode]{python}{linenos,fontsize=\footnotesize,frame=lines,numbersep=4pt}

\newmintedfile[makef]{basemake}{linenos,fontsize=\footnotesize,frame=lines,numbersep=4pt}
\definecolor{bg}{RGB}{22,43,58}
\newminted[shell]{console}{linenos=false,fontsize=\footnotesize,breaklines=true, frame=single} % Linea de comandos
\renewcommand\listingscaption{Código}

\makeatletter
\AtBeginEnvironment{minted}{\dontdofcolorbox}
\def\dontdofcolorbox{\renewcommand\fcolorbox[4][]{##4}}
\makeatother

% uso:
% Ejemplo de uso explícito:
% \begin{py}
% >>> list("abcd")
% ['a', 'b', 'c', 'd']
% \end{py}
% 
% Ahora ejemplo de código en file:
% \pycode{Chapters/intro/code/hola.py}
% 
% También se puede poner un sector del file:
% \pycode[firstline=6, lastline=7]{Chapters/intro/code/hola.py}
% 
% También se puede poner código \textit{inline}: \mip{print('¡Hola mundo!')} y en una sola línea:
% \slp|if __name__ == '__main__')|
% 
% Por último, se puede poner el código en un entorno \textit{float}, esto es, como las tablas y las figuras, con un caption y un label para luego hacer referencias, como por ejemplo al Código \ref{code:hola}.


\usepackage{tikz}
\usetikzlibrary{shapes,shadows,arrows,positioning,matrix,chains,backgrounds,fit}

\tikzset{
    %Define standard arrow tip
    >=stealth',
    %Define style for boxes
    obj/.style={
           rectangle,
           rounded corners,
           draw, very thick,
           text width=10em, fill=green!20,
           minimum height=2em,
           text centered, drop shadow},
    proc/.style={
	    rectangle, rounded corners,
	    draw,fill=red!50,very thick,
	    text width=8em,minimum height=2em,
	    text centered, drop shadow},
    % Define arrow style
    pil/.style={
           ->,
           thick,
           shorten <=2pt,
           shorten >=2pt,}
}

\setbeamertemplate{bibliography item}{%
  \ifboolexpr{ test {\ifentrytype{book}} or test {\ifentrytype{mvbook}}
    or test {\ifentrytype{collection}} or test {\ifentrytype{mvcollection}}
    or test {\ifentrytype{reference}} or test {\ifentrytype{mvreference}} }
    {\setbeamertemplate{bibliography item}{\faBook}}
    {\ifentrytype{online}
            {\setbeamertemplate{bibliography item}{\faGlobe}}
   {\setbeamertemplate{bibliography item}{\faFileText}}}%
  \usebeamertemplate{bibliography item}}

\defbibenvironment{bibliography}
  {\list{}
     {\settowidth{\labelwidth}{\usebeamertemplate{bibliography item}}%
      \setlength{\leftmargin}{\labelwidth}%
      \setlength{\labelsep}{\biblabelsep}%
      \addtolength{\leftmargin}{\labelsep}%
      \setlength{\itemsep}{\bibitemsep}%
      \setlength{\parsep}{\bibparsep}}}
  {\endlist}
  {\item}
\newcommand{\bcite}[1]{\citeauthor{#1}, \citetitle{#1} (\citeyear{#1})}


\title{Errores en el cálculo numérico}
\subtitle{}

\begin{document}
\maketitle

\begin{frame}{Cálculo numérico}
\begin{columns}[t]
\cx
\textbf{¿Para qué?}

\begin{itemize}
    \item Análisis numérico
    \item Manipulación simbólica
    \item Colección y análisis de datos
    \item Visualización 
    \item Simulación
\end{itemize}
\pause

\cx
\textbf{¿Cómo?}

\begin{itemize}
    \item Modelado: sistema de ecuaciones, ecuaciones diferenciales, integral
    \item Método numérico: elección, parametrización, \alert{estimación de errores}
    \item Programación: \alert{Python}, C/C++, Fortran, Julia
    \item Ejecución del código
    \item Interpretación de resultados: visualización, análisis estadístico, rediseño y ejecución
\end{itemize}
\end{columns}
\end{frame}

\begin{frame}{Errores}
    \begin{enumerate}
    \item \textbf{Error inherente}. Provienen desde el principio en los datos originales y están fuera del alcance del control de cálculo. Ejemplo: incertezas en las mediciones. \medskip
    \item \textbf{Error de truncamiento}. Se producen como resultado de reemplazar un proceso infinito por uno finito. Ejemplo: usar solo los primeros términos de una serie de Taylor. \medskip
    \item \textbf{Error de redondeo}. Se originan en la representación con precisión finita de los números en una computadora. \medskip
    \item \textbf{Error por equivocación}. Causado por realizar una operación aritmética incorrectamente.
    \end{enumerate} 

    \par\noindent\rule{\textwidth}{0.5pt} \pause

\begin{center}
    1 -- 3 son errores \textbf{inevitables} $\mapsto$ \alert{control del error}.

    4 es \textbf{evitable}.
\end{center}
\end{frame}

\begin{frame}[fragile]{Números enteros y de punto flotante}
\begin{columns}[t]
\cx
Una cuenta simple: $a + a + a - 3a = 0$
\begin{py}
>>> 0.1 + 0.1 + 0.1 - 0.3
5.551115123125783e-17
\end{py}
En computadoras hay \textbf{dos tipos} de números:
\begin{itemize}
    \item \textbf{Punto fijo:} cantidad fija de números decimales\\
        Ejemplo: $35.6247, 0.4573, -1.0000$ \\
        Enteros: 0 números decimales $\mapsto$ \textbf{exacta}
    \item \textbf{Punto flotante:} cantidad fija de cifras significativas\\
        Ejemplo: $0.1900 \cdot 10^4, 0.8691 \cdot 10^{-6}, -0.2000 \cdot 10^{-13}$ (4 cifras significativas) $\mapsto$ representación \textbf{aproximada}
\end{itemize}
\begin{py}
>>> 7 + 0.000000000000001
7.000000000000001
>>> 7 + 0.0000000000000001
7.0
>>> 0.1 + 0.2
0.30000000000000004
\end{py}

\cx
\textbf{Estándar IEEE 754}: tres \textbf{enteros}

\[ r = (-1)^s \, m \, b^e \]

$s$: signo, $m$: mantisa, $b$: base (10, 2, 16), $e$: exponente.

Ejemplo decimal:
\begin{align*}
    s &= 0 \\
    m &= 31415926 \\
    b &= 10 \\
    e &= -7 \\
    r &= (-1)^0 31415926 \cdot 10^{-7} = 3.1415926
\end{align*}

Ejemplo binario: 
\[ r = \pm m \cdot 2^n, \quad m = 0.d_1d_2\cdots d_k, \quad d_1 > 0 \]
número de máquina de $k$-dígitos.
\end{columns}
\end{frame}

\begin{frame}
\begin{columns}[t]
\cx
\textbf{Características del formato:}
\begin{itemize}
    \item Permite representar números de órdenes de magnitud enormemente dispares (limitado por la longitud del exponente)
    \item Proporciona la misma precisión relativa para todos los órdenes (limitado por la longitud de la mantisa)
    \item Permite cálculos entre magnitudes (número grande $\times$ número pequeño) conservando la precisión de ambos en el resultado.
\end{itemize}

\begin{center}
    $6.022 \cdot 10^{23} \longleftrightarrow$ \texttt{6.022E23}
\end{center}

\cx
\begin{itemize}
    \item Existe solo un número finito de números de máquina, y son menos ``densos'' a medida que el número es más grande. Hay tantos números entre 2 y 4 como entre 1024 y 2048.
    \item El menor número de máquina positivo $\varepsilon_m$ para el cual $1 + \varepsilon_m > 1$ se denomina \textbf{precisión de la máquina}. No se pueden representar números en los intervalos $[1, 1 + \varepsilon_m], [2, 2 + 2 \varepsilon_m], \cdots $
\end{itemize}

\end{columns}
\end{frame}

\section*{Bibliografía}
\begin{frame}[allowframebreaks]{Lecturas recomendadas}
\begin{itemize}
 \item \fullcite{kreyszig2011}. Capítulo 6.
 \item \fullcite{oneil2012es}. Capítulo 1.
 \item \fullcite{stroud2020}. Programas 2, 3 y 4.
\end{itemize}
\end{frame}

\end{document}

