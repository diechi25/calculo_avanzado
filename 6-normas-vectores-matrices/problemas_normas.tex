\documentclass[11pt]{article}

\usepackage{exsheets}
\usepackage[paper=a4paper, headheight=110pt,showframe=false, 
            layoutvoffset=2em,
            bottom=2cm, top=3.5cm]{geometry}

\usepackage[spanish]{babel}
\usepackage[babel]{microtype}
\usepackage{lipsum}
\usepackage{unicode-math}
% Fonts can be customized here.
\defaultfontfeatures{Mapping=tex-text}
\setmainfont [Ligatures={Common}]{Linux Libertine O}
\setmonofont[Scale=0.9]{Linux Libertine Mono O}
%\usepackage[svgnames]{xcolor} % Gestión de colores
\usepackage{hyperref}
\hypersetup{
  colorlinks=true, linktocpage=true, pdfstartpage=3, pdfstartview=FitV,%
  breaklinks=true, pageanchor=true,%
  pdfpagemode=UseNone, %
  plainpages=false, bookmarksnumbered, bookmarksopen=true, bookmarksopenlevel=1,%
  hypertexnames=true, pdfhighlight=/O,%nesting=true,%frenchlinks,%
  urlcolor=Maroon, linkcolor=RoyalBlue, citecolor=Blue, %pagecolor=RoyalBlue,%
  pdftitle={},%
  pdfauthor={\textcopyright\ C. Manuel Carlevaro},%
  pdfsubject={},%
  pdfkeywords={},%
  pdfcreator={XeLaTeX},%
  pdfproducer={XeLaTeX}%
}

\usepackage{hyperref}
\usepackage{fontawesome, gensymb}
\usepackage{graphicx}
\setlength{\parindent}{3em}
\setlength{\parskip}{1em} 
\usepackage[shortlabels]{enumitem}

%\NewDocumentCommand{\evalat}{sO{\big}mm}{%
  %\IfBooleanTF{#1}
   %{\mleft. #3 \mright|_{#4}}
   %{#3#2|_{#4}}%
%}


\title{Cálculo avanzado}
\author{Dpto. de Ingenería Mecánica}
\date{Clase 8: Norma de vectores y matrices}


\begin{document}
% \maketitle

\begin{center}
\framebox[1.0\textwidth][c]{
\huge{\textsc{Cálculo Avanzado}} 
}
\end{center} 

\begin{center}
\vspace{\baselineskip}
\Large{\textsc{Departamento de Ingenería Mecánica}} \\
\textsc{Facultad Regional La Plata} \\
\textsc{Universidad Tecnológica Nacional}
\end{center}

% \vspace{1em}

\begin{center}
\begin{tabular}{r l}
    \textbf{Práctica:} & 8 \\
 \textbf{Tema:} & Norma de vectores y matrices. \\
 \textbf{Profesor Titular:} & Manuel Carlevaro \\
 \textbf{Jefe de Trabajos Prácticos:} & Diego Amiconi \\
\end{tabular}\end{center}

\vspace{1em}

\begin{question} % Burden y Faires - Exercise set 7.1 p1 pg 441
    Halle las normas $l_1$, $l_2$ y $l_{\infty}$ de los siguientes vectores:
    \begin{enumerate}[a)]
    \item $\bm{x} = [3, -4, 0, 3/2]$
    \item $\bm{x} = [2, 1, -3, 4]$
    \item $\bm{x} = [\sen k, \cos k, 2^k]$ para un entero positivo fijo $k$.
    \end{enumerate}
\end{question}

\begin{question} % Burden y Faires - Exercise set 7.1 p4 pg 441
    Halle las normas por componentes $\lVert \cdot \rVert_F$, $\lVert \cdot \rVert_{\text{máx}}$ y  $p = 1$ de las siguientes matrices:
    \begin{enumerate}[a)]
        \item $\bm{A} = \begin{bmatrix} 10 & 15 \\ 0 & 1 \end{bmatrix}$
        \item $\bm{A} = \begin{bmatrix}
            10 & 0 \\ 15 & 1 \end{bmatrix}$
        \item $\bm{A} = \begin{bmatrix}
            2 & -1 & 0 \\ -1 & 2 & -1 \\
            0 & -1 & 2\end{bmatrix}$
        \item $\bm{A} = \begin{bmatrix}
            4 & -1 & 7 \\ -1 & 4 & 0 \\
            -7 & 0 & 4\end{bmatrix}$
    \end{enumerate}
\end{question}

\begin{question} % Quarteroni, pg 41
    Mostrar que la norma por componente $\lVert \cdot \rVert_{\text{máx}}$ no satisface la propiedad sub-multiplicativa mediante un contraejemplo.
\end{question}

\begin{question} % Burden y Faires - Exercise set 7.1 p4 pg 441
    Para las matrices del ejercicio 2, calcule las normas inducidas $\lVert \cdot \rVert_{1}$ y  $\lVert \cdot \rVert_{\infty}$.
\end{question}

\end{document}
