\documentclass[11pt]{article}

\usepackage{exsheets}
\usepackage[paper=a4paper, headheight=110pt,showframe=false, 
            layoutvoffset=2em,
            bottom=2cm, top=3.5cm]{geometry}

\usepackage[spanish]{babel}
\usepackage[babel]{microtype}
\usepackage{lipsum}
\usepackage{unicode-math}
% Fonts can be customized here.
\defaultfontfeatures{Mapping=tex-text}
\setmainfont [Ligatures={Common}]{Linux Libertine O}
\setmonofont[Scale=0.9]{Linux Libertine Mono O}
%\usepackage[svgnames]{xcolor} % Gestión de colores
\usepackage{hyperref}
\hypersetup{
  colorlinks=true, linktocpage=true, pdfstartpage=3, pdfstartview=FitV,%
  breaklinks=true, pageanchor=true,%
  pdfpagemode=UseNone, %
  plainpages=false, bookmarksnumbered, bookmarksopen=true, bookmarksopenlevel=1,%
  hypertexnames=true, pdfhighlight=/O,%nesting=true,%frenchlinks,%
  urlcolor=Maroon, linkcolor=RoyalBlue, citecolor=Blue, %pagecolor=RoyalBlue,%
  pdftitle={},%
  pdfauthor={\textcopyright\ C. Manuel Carlevaro},%
  pdfsubject={},%
  pdfkeywords={},%
  pdfcreator={XeLaTeX},%
  pdfproducer={XeLaTeX}%
}

\usepackage{hyperref}
\usepackage{fontawesome, gensymb}
\usepackage{graphicx}
\setlength{\parindent}{3em}
\setlength{\parskip}{1em} 
\usepackage[shortlabels]{enumitem}

%\NewDocumentCommand{\evalat}{sO{\big}mm}{%
  %\IfBooleanTF{#1}
   %{\mleft. #3 \mright|_{#4}}
   %{#3#2|_{#4}}%
%}


\title{Cálculo avanzado}
\author{Dpto. de Ingenería Mecánica}
\date{Clase 9: Autovalores y autovectores}


\begin{document}
% \maketitle

\begin{center}
\framebox[1.0\textwidth][c]{
\huge{\textsc{Cálculo Avanzado}} 
}
\end{center} 

\begin{center}
\vspace{\baselineskip}
\Large{\textsc{Departamento de Ingenería Mecánica}} \\
\textsc{Facultad Regional La Plata} \\
\textsc{Universidad Tecnológica Nacional}
\end{center}

% \vspace{1em}

\begin{center}
\begin{tabular}{r l}
    \textbf{Práctica:} & 9 \\
 \textbf{Tema:} & Autovalores y autovectores I. \\
 \textbf{Profesor Titular:} & Manuel Carlevaro \\
 \textbf{Jefe de Trabajos Prácticos:} & Diego Amiconi \\
 \textbf{Ayudante de Primera:} & Lucas Basiuk 
\end{tabular}\end{center}

\vspace{1em}
\begin{question} % Burden Faires Ejemplo 3 pg. 434
    Muestre que se puede formar una base para $\mathbb{R}^3$ usando los autovalores de la matriz
    \[ \bm{A} = \begin{bmatrix} 2 & 0 & 0 \\
        1 & 1 & 2 \\
    1 & -1 & 4 \end{bmatrix} \]
\end{question}

\begin{question} % Burden Faires Ejemplo 4 pg. 434
Muestre que ningún conjunto de autovectores de la matriz $3 \times 3$
\[ \bm{B} = \begin{bmatrix} 2 & 1 & 0 \\
           0 & 2 & 0 \\
       0 & 0 & 3 \end{bmatrix} \]
       puede formar una base en $\mathbb{R}^3$.
\end{question}

\begin{question} % Burden Faires Ejemplo 5, pg 435
    \begin{enumerate}[a)]
        \item Muestre que los vectores $\bm{v}_1 = [0, 4, 2]^{\intercal}$ $\bm{v}_2 = [-5, -1, 2]^{\intercal}$ y $\bm{v}_3 = [1, -1, 2]^{\intercal}$ forman un conjunto ortogonal.
        \item Use el conjunto anterior para formar una base ortonormal.
    \end{enumerate}
\end{question}


\begin{question} % Burden Faires Ejemplo 6, pg 435
Use el proceso de Gram-Schmidt para determinar un conjunto de vectores ortogonales a partir de los vectores linealmente independientes:
\[ \bm{x}_1 = [1, 0, 0]^{\intercal}, \; \bm{x}_2 = [1, 1, 0]^{\intercal}, \; \bm{x}_3 = [1, 1, 1]^{\intercal} \]
\end{question}

\begin{question} % Burden Faires Ejemplo 1, pg 437
Muestre que la matriz
\[ \bm{Q} = [\bm{u}_1, \bm{u}_2, \bm{u}_3] = 
    \begin{bmatrix} 0 & -\frac{\sqrt{30}}{6} & \frac{\sqrt{6}}{6} \\[0.3em]
        \frac{2 \sqrt{5}}{5} & -\frac{\sqrt{30}}{30} & -\frac{\sqrt{6}}{6} \\[0.3em]
    \frac{\sqrt{5}}{5} & \frac{\sqrt{30}}{15} & \frac{\sqrt{6}}{3} \end{bmatrix} \]
    formada a partir del conjunto ortonormal de vectores encontrado en el problema 3, es una matriz ortogonal.
\end{question}

\end{document}
