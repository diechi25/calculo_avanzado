\documentclass[11pt]{article}

\usepackage{exsheets}
\usepackage[paper=a4paper, headheight=110pt,showframe=false, 
            layoutvoffset=2em,
            bottom=2cm, top=3.5cm]{geometry}

\usepackage[spanish]{babel}
\usepackage[babel]{microtype}
\usepackage{lipsum}
\usepackage{unicode-math}
% Fonts can be customized here.
\defaultfontfeatures{Mapping=tex-text}
\setmainfont [Ligatures={Common}]{Linux Libertine O}
\setmonofont[Scale=0.9]{Linux Libertine Mono O}
%\usepackage[svgnames]{xcolor} % Gestión de colores
\usepackage{hyperref}
\hypersetup{
  colorlinks=true, linktocpage=true, pdfstartpage=3, pdfstartview=FitV,%
  breaklinks=true, pageanchor=true,%
  pdfpagemode=UseNone, %
  plainpages=false, bookmarksnumbered, bookmarksopen=true, bookmarksopenlevel=1,%
  hypertexnames=true, pdfhighlight=/O,%nesting=true,%frenchlinks,%
  urlcolor=Maroon, linkcolor=RoyalBlue, citecolor=Blue, %pagecolor=RoyalBlue,%
  pdftitle={},%
  pdfauthor={\textcopyright\ C. Manuel Carlevaro},%
  pdfsubject={},%
  pdfkeywords={},%
  pdfcreator={XeLaTeX},%
  pdfproducer={XeLaTeX}%
}

\usepackage{hyperref}
\usepackage{fontawesome, gensymb}
\usepackage{graphicx}
\setlength{\parindent}{3em}
\setlength{\parskip}{1em} 
\usepackage[shortlabels]{enumitem}

%\NewDocumentCommand{\evalat}{sO{\big}mm}{%
  %\IfBooleanTF{#1}
   %{\mleft. #3 \mright|_{#4}}
   %{#3#2|_{#4}}%
%}


\title{Cálculo avanzado}
\author{Dpto. de Ingenería Mecánica}
\date{Unidad 10: Problemas de valores iniciales}


\begin{document}
% \maketitle

\begin{center}
\framebox[1.0\textwidth][c]{
\huge{\textsc{Cálculo Avanzado}} 
}
\end{center} 

\begin{center}
\vspace{\baselineskip}
\Large{\textsc{Departamento de Ingenería Mecánica}} \\
\textsc{Facultad Regional La Plata} \\
\textsc{Universidad Tecnológica Nacional}
\end{center}

% \vspace{1em}

\begin{center}
\begin{tabular}{r l}
    \textbf{Práctica:} & Unidad 10 \\
 \textbf{Tema:} & Problemas de valores iniciales. \\
 \textbf{Profesor Titular:} & Manuel Carlevaro \\
 \textbf{Jefe de Trabajos Prácticos:} & Diego Amiconi \\
 \textbf{Ayudante de Primera:} & Lucas Basiuk 
\end{tabular}\end{center}

\vspace{1em}

\begin{question} % James Epperson p. 5 pg 335 (solución pg 213)
Para cada problema de valor inicial siguiente, determine la constante de Lipschitz $K$ en el dominio dado.
\begin{enumerate}[a)]
    \item $y' = 1 - 3y, y(0) = 0, D = \{(t, y) \, | \, -1 \leq t \leq 1; 0 \leq y \leq 2 \}$;
    \item $y' = y(1-y), y(0) = 1/2, D = (-1, 1) \times (0, 2)$;
    \item $y' = y^2, y(0) = 1, D = (-1, 1) \times (0, 2)$.
\end{enumerate}
\end{question}

\begin{question} % James Epperson p. 1 pg 338 (solución pg 213)
Use el método de Euler con $h = 1/4$ para calcular aproximadamente los valores de $y(1)$ para cada problema de valor inicial siguiente. Realizar los cálculos, sin un programa de computadora, para producir una tabla ordenada de pares $(t_k, y_k)$.
\begin{enumerate}[a)]
    \item $y' = y(1-y), y(0) = 1/2$;
    \item $t y' = y(\sen t), y(0) = 2$;
    \item $y' = y(1 + e^{2 t}), y(0) = 1$;
    \item $y' + 2 y = 1, y(0) = 2$.
\end{enumerate}
\end{question}


\begin{question} % James Epperson p. 3 pg 338 (solución pg 214)
Escriba un programa en Python que resuelva cada uno de los problemas de valor inicial del Ejercicio 2, utilizando el método de Euler con un paso $h = 1/16$.
\end{question}


\begin{question} % James Epperson p. 4 pg 338 (solución pg 213)
Para cada uno de los problemas de valor inicial siguiente, aproxime la solución utilizando el método de Euler con una secuencia de pasos decreciente $h^{-1} = 2, 4, 8, \ldots$. Para los problemas en que se da la solución exacta, compare la precisión alcanzada en el intervalo $[0, 1]$ con la precisión teórica.
\begin{enumerate}[a)]
    \item $y' + 4 y = 1, y(0) = 1; y(t) = \tfrac{1}{4}(3 e^{-4t} + 1)$;
    \item $y'= -y \ln y, y(0) = 3; y(t) = e^{(\ln 3) e^{-t}}$;
    \item $y' + y = \sen 4 \pi t, y(0) = 1/2$;
    \item $y' + \sen y = 0, y(0) = 1$.
\end{enumerate}
\end{question}


\begin{question} % James Epperson p. 1 pg 365 (solución pg 225)
Utilize el método de Runge-Kutta de segundo orden para resolver el Ejercicio 2.
\end{question}

\begin{question} % James Epperson p. 2 pg 365 (solución pg 226)
Repita el Ejercicio anterior usando el método de Runge-Kutta de cuarto orden. Compare la precisión obtenida con los resultados generados con los métodos anteriores.
\end{question}

\begin{question} % James Epperson p. 4 pg 365 (solución pg 227)
Escriba un programa en Python que resuelva los problemas con valores iniciales del Ejercicio 2 utilizando el método de Runge-Kutta de cuarto orden con $h = 1/16$. Compare los resultados obtenidos con los del Ejercicio 3.
\end{question}

\begin{question} % James Epperson p. 6 pg 365
Repita el Ejercicio 4, pero utilizando ahora el método de Runge-Kutta de cuarto orden.
\end{question}

\end{document}
