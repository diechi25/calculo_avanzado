\documentclass[11pt]{article}

\usepackage{exsheets}
\usepackage[paper=a4paper, headheight=110pt,showframe=false, 
            layoutvoffset=2em,
            bottom=2cm, top=3.5cm]{geometry}

\usepackage[spanish]{babel}
\usepackage[babel]{microtype}
\usepackage{lipsum}
\usepackage{unicode-math}
% Fonts can be customized here.
\defaultfontfeatures{Mapping=tex-text}
\setmainfont [Ligatures={Common}]{Linux Libertine O}
\setmonofont[Scale=0.9]{Linux Libertine Mono O}
%\usepackage[svgnames]{xcolor} % Gestión de colores
\usepackage{hyperref}
\hypersetup{
  colorlinks=true, linktocpage=true, pdfstartpage=3, pdfstartview=FitV,%
  breaklinks=true, pageanchor=true,%
  pdfpagemode=UseNone, %
  plainpages=false, bookmarksnumbered, bookmarksopen=true, bookmarksopenlevel=1,%
  hypertexnames=true, pdfhighlight=/O,%nesting=true,%frenchlinks,%
  urlcolor=Maroon, linkcolor=RoyalBlue, citecolor=Blue, %pagecolor=RoyalBlue,%
  pdftitle={},%
  pdfauthor={\textcopyright\ C. Manuel Carlevaro},%
  pdfsubject={},%
  pdfkeywords={},%
  pdfcreator={XeLaTeX},%
  pdfproducer={XeLaTeX}%
}

\usepackage{hyperref}
\usepackage{fontawesome, gensymb}
\usepackage{graphicx}
\setlength{\parindent}{3em}
\setlength{\parskip}{1em} 
\usepackage[shortlabels]{enumitem}

%\NewDocumentCommand{\evalat}{sO{\big}mm}{%
  %\IfBooleanTF{#1}
   %{\mleft. #3 \mright|_{#4}}
   %{#3#2|_{#4}}%
%}


\title{Cálculo avanzado}
\author{Dpto. de Ingenería Mecánica}
\date{Clase 3: series de Fourier}


\begin{document}
% \maketitle

\begin{center}
\framebox[1.0\textwidth][c]{
\huge{\textsc{Cálculo Avanzado}} 
}
\end{center} 

\begin{center}
\vspace{\baselineskip}
\Large{\textsc{Departamento de Ingenería Mecánica}} \\
\textsc{Facultad Regional La Plata} \\
\textsc{Universidad Tecnológica Nacional}
\end{center}

% \vspace{1em}

\begin{center}
\begin{tabular}{r l}
    \textbf{Práctica:} & 3 \\
 \textbf{Tema:} & Funciones ortogonales. Series de Fourier. \\
 \textbf{Profesor Titular:} & Manuel Carlevaro \\
 \textbf{Jefe de Trabajos Prácticos:} & Diego Amiconi \\
 \textbf{Ayudante de Primera:} & Lucas Basiuk 
\end{tabular}\end{center}

\vspace{1em}

\begin{question} % H. Gross 3.8.1(L) pg. 4
 Mostrar que el conjunto de funciones
 \[ \{1 , \cos x, \cos 2x, \ldots, \cos nx, \ldots, \sen x, \sen 2x, \ldots, \sen nx, \ldots \} \]
 es ortogonal en el intervalo $[-\pi, \pi]$.
\end{question}

\begin{question} % H. Gross 3.8.2(L) pg. 4
Suponga que $f(x)$ es integrable en $[-\pi, \pi]$ y sea
\[ F(x) = \sum_{n = 0}^{\infty} a_n \cos nx + \sum_{n = 1}^{\infty} a_n \sen nx \]
la representación de Fourier de $f(x)$ relativa a $\{1, \cos x, \cos 2x, \ldots, \sen x, \sen 2x, \ldots \}$. Usar el método descripto en la clase para determinar $a_n$ y $b_n$.
\end{question}

\begin{question} % H. Gross 3.8.3(L) pg. 4
Definiendo $f(x)$ en $[-\pi, \pi]$ como
\[ f(x) = \begin{cases}
    -1,&  -\pi < x < 0 \\
    \phantom{-}1,&  0 < x < \pi
        \end{cases} \]
\begin{enumerate}[a)]
\item Derive la representación de Fourier de $f(x)$.
\item Use la parte a) para evaluar la suma
    \[ \sum_{n = 0}^{\infty} \frac{(-1)^n}{2 n + 1} \]
\end{enumerate}
\end{question}

\begin{question} % H. Gross 3.8.4(L) pg. 4
    \begin{enumerate}[ a) ]
    \item Halle la representación de Fourier de $f(x) = |x|, \; -\pi \leq x \leq \pi$.
    \item Use el resultado de la parte a) para computar
        \[ \sum_{n = 0}^{\infty} \frac{1}{(2 n + 1)^2} \]
\end{enumerate}
\end{question}

\begin{question} % H. Gross 3.8.5(L) pg. 5
\begin{enumerate}[a)]
\item Encuentre la expansión en serie de Fourier de la función
    \[ f(x) = 
        \begin{cases}
            0,& -\pi < x < 0 \\
            x^2,& 0 \leq x \< \pi
        \end{cases} \]
\item Dibuje $y = F(x)$ donde $F$ es la representación de Fourier de $f$.
\item Use a) para evaluar
    \[ \sum_{n = 1}^{\infty} \frac{1}{n^2} \]
\end{enumerate}
\end{question}

\begin{question} % H. Gross 3.8.7(L) pg. 6
Sea $f$ definida por $f(x) = x, \; -1 < x < 1$. Exprese $f$ como una serie trigonométrica.
\end{question}

\end{document}
