\metroset{titleformat=smallcaps,block=fill,numbering=counter,progressbar=frametitle,sectionpage=none}
\setbeamersize{text margin left=5mm,text margin right=5mm} 
% %%%%%%%%%%%%%%%%%%%%%%%%%%%%%%%%%%%%%%%%%%%%%%%%%%%%%%%%%%%%%%%%%%%%%%%%%%%%%%
% \embedvideo{<poster or text>}{<video file (MP4+H264)>}
% \embedvideo*{...}{...}                     % auto-play
%%%%%%%%%%%%%%%%%%%%%%%%%%%%%%%%%%%%%%%%%%%%%%%%%%%%%%%%%%%%%%%%%%%%%%%%%%%%%%

\usepackage[bigfiles]{pdfbase}
\ExplSyntaxOn
\NewDocumentCommand\embedvideo{smm}{
  \group_begin:
  \leavevmode
  \tl_if_exist:cTF{file_\file_mdfive_hash:n{#3}}{
    \tl_set_eq:Nc\video{file_\file_mdfive_hash:n{#3}}
  }{
    \IfFileExists{#3}{}{\GenericError{}{File~`#3'~not~found}{}{}}
    \pbs_pdfobj:nnn{}{fstream}{{}{#3}}
    \pbs_pdfobj:nnn{}{dict}{
      /Type/Filespec/F~(#3)/UF~(#3)
      /EF~<</F~\pbs_pdflastobj:>>
    }
    \tl_set:Nx\video{\pbs_pdflastobj:}
    \tl_gset_eq:cN{file_\file_mdfive_hash:n{#3}}\video
  }
  %
  \pbs_pdfobj:nnn{}{dict}{
    /Type/RichMediaInstance/Subtype/Video
    /Asset~\video
    /Params~<</FlashVars (
      source=#3&
      skin=SkinOverAllNoFullNoCaption.swf&
      skinAutoHide=true&
      skinBackgroundColor=0x5F5F5F&
      skinBackgroundAlpha=0
    )>>
  }
  %
  \pbs_pdfobj:nnn{}{dict}{
    /Type/RichMediaConfiguration/Subtype/Video
    /Instances~[\pbs_pdflastobj:]
  }
  %
  \pbs_pdfobj:nnn{}{dict}{
    /Type/RichMediaContent
    /Assets~<<
      /Names~[(#3)~\video]
    >>
    /Configurations~[\pbs_pdflastobj:]
  }
  \tl_set:Nx\rmcontent{\pbs_pdflastobj:}
  %
  \pbs_pdfobj:nnn{}{dict}{
    /Activation~<<
      /Condition/\IfBooleanTF{#1}{PV}{XA}
      /Presentation~<</Style/Embedded>>
    >>
    /Deactivation~<</Condition/PI>>
  }
  %
  \hbox_set:Nn\l_tmpa_box{#2}
  \tl_set:Nx\l_box_wd_tl{\dim_use:N\box_wd:N\l_tmpa_box}
  \tl_set:Nx\l_box_ht_tl{\dim_use:N\box_ht:N\l_tmpa_box}
  \tl_set:Nx\l_box_dp_tl{\dim_use:N\box_dp:N\l_tmpa_box}
  \pbs_pdfxform:nnnnn{1}{1}{}{}{\l_tmpa_box}
  %
  \pbs_pdfannot:nnnn{\l_box_wd_tl}{\l_box_ht_tl}{\l_box_dp_tl}{
    /Subtype/RichMedia
    /BS~<</W~0/S/S>>
    /Contents~(embedded~video~file:#3)
    /NM~(rma:#3)
    /AP~<</N~\pbs_pdflastxform:>>
    /RichMediaSettings~\pbs_pdflastobj:
    /RichMediaContent~\rmcontent
  }
  \phantom{#2}
  \group_end:
}
\ExplSyntaxOff
%%%%%%%%%%%%%%%%%%%%%%%%%%%%%%%%%%%%%%%%%%%%%%%%%%%%%%%%%%%%%%%%%%%%%%%%%%%%%%

\usepackage{fontspec,minted}
\usepackage[scale=1]{ccicons}
\usepackage{metalogo}
\usepackage{xcolor,colortbl}
\usepackage{multicol,multirow,booktabs}
\usepackage{appendixnumberbeamer}
\usepackage{graphicx}
\usepackage{mismath}
\usepackage{bm}
\usepackage{fontawesome5}
\usepackage{csquotes}
\usepackage[backend=biber, natbib, sorting=nyt, doi=true, url=false, url=false, isbn=false, maxbibnames=10]{biblatex}
\addbibresource{../utils/refs.bib}

\usepackage[spanish, es-nodecimaldot]{babel}
\deftranslation[to=spanish]{Definition}{Definición}
\deftranslation[to=spanish]{Theorem}{Teorema}
\deftranslation[to=spanish]{Example}{Ejemplo}

\usepackage{mathtools, mathrsfs}
\usefonttheme{professionalfonts}
\usepackage{textcomp, wasysym}

\setsansfont[BoldFont={Iwona Heavy}, Numbers={Lining, Proportional}]{Iwona Light}
% \setmathsfont(Digits)[Numbers={Lining, Proportional}]{Fira Sans Light}
\setmonofont[Scale=MatchLowercase]{DejaVu Sans Mono}

\setbeamercolor{alerted text}{fg=red,bg=black!2}
\setbeamercolor{progress bar}{fg=red,bg=red!2}
\setbeamertemplate{itemize item}{\faCaretRight}
\setbeamertemplate{itemize subitem}{ \faAngleRight}
\setbeamertemplate{blocks}[shadow=false]
\setbeamercolor{block title}{bg=black!30,fg=red}
\setbeamercolor{block body}{bg=black!20,fg=black}
\setbeamertemplate{theorem begin}
{%
\begin{\inserttheoremblockenv}
{%
\inserttheoremheadfont
%{Teorema:}
\inserttheoremname
\ifx\inserttheoremaddition\@empty\else\ : \inserttheoremaddition\fi%
\inserttheorempunctuation
}%
}
\setbeamertemplate{theorem end}{\end{\inserttheoremblockenv}}
\makeatother


 
\usepackage{gensymb,amssymb}
\usepackage{siunitx}
\DeclareSIUnit{\nada}{\relax}
\usepackage{upquote}
\usepackage{cancel}
\usepackage{algpseudocode}
\algrenewcommand\algorithmicrequire{\textbf{Requiere}}
\algrenewcommand\algorithmicensure{\textbf{Devuelve}}
\setbeamertemplate{blocks}[shadow=false]

\newcommand{\cx}{\column{0.5\textwidth}}
\newcommand{\cw}[1]{\column{#1\textwidth}}

\author{Manuel Carlevaro}
\date{{\tiny Departamento de Ingeniería Mecánica \\
             Grupo de Materiales Granulares - UTN FRLP \\
             manuel.carlevaro@gmail.com }}
\institute{
  \vspace{6em}
  \centering
  {\tiny
  Cálculo Avanzado \enspace • \enspace 2024 \\
    \faLinux \- $\cdot$ \- \fontspec{TeX Gyre Pagella}\XeLaTeX \- $\cdot$ \- \ccbysa }
}

%% Operadores
\DeclareMathOperator{\sen}{sen}
\DeclareMathOperator{\senc}{senc}
\DeclareMathOperator{\sign}{sign}
\newcommand{\T}[1]{\underline{\bm{#1}}}
\DeclareMathOperator{\Tr}{Tr}
\DeclareMathOperator{\rg}{rg}
\DeclareMathOperator{\cond}{cond}

\usepackage{hyperref}
\hypersetup{
    colorlinks,
    citecolor=blue,
    filecolor=black,
    linkcolor=blue,
    urlcolor=blue
}
\urlstyle{same}

%% Códigos
\usepackage{minted}
\newminted[cpp]{cpp}{linenos,fontsize=\footnotesize,frame=lines,numbersep=4pt}
\newmintedfile[cppcode]{cpp}{linenos,fontsize=\footnotesize,frame=lines,numbersep=4pt}
\newcommand{\mic}[1]{\mintinline{C++}{#1}}

\newminted[py]{python}{linenos,fontsize=\footnotesize,frame=lines,numbersep=4pt}
\newminted[pyc]{pycon}{linenos,fontsize=\footnotesize,frame=lines,numbersep=4pt} % Consola de Python
\newminted[ipy3]{ipython3}{linenos,fontsize=\footnotesize,frame=lines,numbersep=4pt} % Consola de iPython3
\newmintedfile[pycode]{python}{linenos,fontsize=\footnotesize,frame=lines,numbersep=4pt}

\newmintedfile[makef]{basemake}{linenos,fontsize=\footnotesize,frame=lines,numbersep=4pt}
\definecolor{bg}{RGB}{22,43,58}
\newminted[shell]{console}{linenos=false,fontsize=\footnotesize,breaklines=true, frame=single} % Linea de comandos
\renewcommand\listingscaption{Código}

\makeatletter
\AtBeginEnvironment{minted}{\dontdofcolorbox}
\def\dontdofcolorbox{\renewcommand\fcolorbox[4][]{##4}}
\makeatother

% uso:
% Ejemplo de uso explícito:
% \begin{py}
% >>> list("abcd")
% ['a', 'b', 'c', 'd']
% \end{py}
% 
% Ahora ejemplo de código en file:
% \pycode{Chapters/intro/code/hola.py}
% 
% También se puede poner un sector del file:
% \pycode[firstline=6, lastline=7]{Chapters/intro/code/hola.py}
% 
% También se puede poner código \textit{inline}: \mip{print('¡Hola mundo!')} y en una sola línea:
% \slp|if __name__ == '__main__')|
% 
% Por último, se puede poner el código en un entorno \textit{float}, esto es, como las tablas y las figuras, con un caption y un label para luego hacer referencias, como por ejemplo al Código \ref{code:hola}.


\usepackage{tikz}
\usetikzlibrary{shapes,shadows,arrows,positioning,matrix,chains,backgrounds,fit}

\tikzset{
    %Define standard arrow tip
    >=stealth',
    %Define style for boxes
    obj/.style={
           rectangle,
           rounded corners,
           draw, very thick,
           text width=10em, fill=green!20,
           minimum height=2em,
           text centered, drop shadow},
    proc/.style={
	    rectangle, rounded corners,
	    draw,fill=red!50,very thick,
	    text width=8em,minimum height=2em,
	    text centered, drop shadow},
    % Define arrow style
    pil/.style={
           ->,
           thick,
           shorten <=2pt,
           shorten >=2pt,}
}

\setbeamertemplate{bibliography item}{%
  \ifboolexpr{ test {\ifentrytype{book}} or test {\ifentrytype{mvbook}}
    or test {\ifentrytype{collection}} or test {\ifentrytype{mvcollection}}
    or test {\ifentrytype{reference}} or test {\ifentrytype{mvreference}} }
    {\setbeamertemplate{bibliography item}{\faBook}}
    {\ifentrytype{online}
            {\setbeamertemplate{bibliography item}{\faGlobe}}
   {\setbeamertemplate{bibliography item}{\faFileText}}}%
  \usebeamertemplate{bibliography item}}

\defbibenvironment{bibliography}
  {\list{}
     {\settowidth{\labelwidth}{\usebeamertemplate{bibliography item}}%
      \setlength{\leftmargin}{\labelwidth}%
      \setlength{\labelsep}{\biblabelsep}%
      \addtolength{\leftmargin}{\labelsep}%
      \setlength{\itemsep}{\bibitemsep}%
      \setlength{\parsep}{\bibparsep}}}
  {\endlist}
  {\item}
\newcommand{\bcite}[1]{\citeauthor{#1}, \citetitle{#1} (\citeyear{#1})}
