\documentclass[11pt]{article}

\usepackage{exsheets}
\usepackage[paper=a4paper, headheight=110pt,showframe=false, 
            layoutvoffset=2em,
            bottom=2cm, top=3.5cm]{geometry}

\usepackage[spanish]{babel}
\usepackage[babel]{microtype}
\usepackage{lipsum}
\usepackage{unicode-math}
% Fonts can be customized here.
\defaultfontfeatures{Mapping=tex-text}
\setmainfont [Ligatures={Common}]{Linux Libertine O}
\setmonofont[Scale=0.9]{Linux Libertine Mono O}
%\usepackage[svgnames]{xcolor} % Gestión de colores
\usepackage{hyperref}
\hypersetup{
  colorlinks=true, linktocpage=true, pdfstartpage=3, pdfstartview=FitV,%
  breaklinks=true, pageanchor=true,%
  pdfpagemode=UseNone, %
  plainpages=false, bookmarksnumbered, bookmarksopen=true, bookmarksopenlevel=1,%
  hypertexnames=true, pdfhighlight=/O,%nesting=true,%frenchlinks,%
  urlcolor=Maroon, linkcolor=RoyalBlue, citecolor=Blue, %pagecolor=RoyalBlue,%
  pdftitle={},%
  pdfauthor={\textcopyright\ C. Manuel Carlevaro},%
  pdfsubject={},%
  pdfkeywords={},%
  pdfcreator={XeLaTeX},%
  pdfproducer={XeLaTeX}%
}

\usepackage{hyperref}
\usepackage{fontawesome, gensymb}
\usepackage{graphicx}
\setlength{\parindent}{3em}
\setlength{\parskip}{1em} 
\usepackage[shortlabels]{enumitem}

%\NewDocumentCommand{\evalat}{sO{\big}mm}{%
  %\IfBooleanTF{#1}
   %{\mleft. #3 \mright|_{#4}}
   %{#3#2|_{#4}}%
%}


\title{Cálculo avanzado}
\author{Dpto. de Ingenería Mecánica}
\date{Clase 4: integral y transformada de Fourier}


\begin{document}
% \maketitle

\begin{center}
\framebox[1.0\textwidth][c]{
\huge{\textsc{Cálculo Avanzado}} 
}
\end{center} 

\begin{center}
\vspace{\baselineskip}
\Large{\textsc{Departamento de Ingenería Mecánica}} \\
\textsc{Facultad Regional La Plata} \\
\textsc{Universidad Tecnológica Nacional}
\end{center}

% \vspace{1em}

\begin{center}
\begin{tabular}{r l}
    \textbf{Práctica:} & 3 \\
 \textbf{Tema:} & Funciones ortogonales. Series de Fourier. \\
 \textbf{Profesor Titular:} & Manuel Carlevaro \\
 \textbf{Jefe de Trabajos Prácticos:} & Diego Amiconi \\
 \textbf{Ayudante de Primera:} & Lucas Basiuk 
\end{tabular}\end{center}

\vspace{1em}

\begin{question} % Kreyszig problem set 11.7 - 1 pg. 517
Halle la representación integral de Fourier de la función $f(x)$ dada por:
\[ f(x) = 
    \begin{cases}
    0 & \text{ si } x < 0 \\
    \dfrac{\pi}{2} & \text{ si } x = 0 \\
    \pi e^{-x} & \text{ si } x > 0 \\
\end{cases} \]
\end{question}


\begin{question} % Kreyszig problem set 11.8 - 1 pg. 522
Obtenga la transformada coseno de Fourier $\hat{f}_c(\omega)$ de la función
\[ f(x) = 
\begin{cases}
    1 & \text{ si } 0 < x < 1 \\
    -1 & \text{ si } 1 < x < 2 \\
    0 & \text{ si } x > 2 
\end{cases} \]
\end{question}


\begin{question} % Kreyszig problem set 11.8 - 11 pg. 522
Obtenga la transformada seno de Fourier $\hat{f}_s(\omega)$ de la función
\[ f(x) = 
\begin{cases}
    x^2 & \text{ si } 0 < x < 1 \\
    0 & \text{ si } x > 1 \\
\end{cases} \]
\end{question}

\begin{question}
Demostrar que la transformada de Fourier es una operación lineal.
\end{question}

\begin{question}
Demostrar que si $f(x)$ es continua en $(-\infty, \infty)$ y $f(x) \rightarrow 0$ cuando $|x| \rightarrow \infty$, y si $f'(x)$ es abolutamente integrable en el eje $x$, entonces:
\[ \mathscr{F}[f'(x)] = i \omega \mathscr{F}[f(x)] \]
\end{question}

\begin{question} % Kreyszig Example 3 pg 527
Encuentre la transformada de Fourier de 
\[ f(x) = x e^{-x^2} \]
\end{question}

\begin{question} % Kreyszig problem set 11.9 - 3 pg. 533
Halle la transformada de Fourier de 
\[ f(x) = 
    \begin{cases}
        1 & \text { si } a < x < b \\
        0 & \text { de otro modo }
    \end{cases}
\]
\end{question}

\end{document}
