\documentclass[11pt]{article}

\usepackage{exsheets}
\usepackage[paper=a4paper, headheight=110pt,showframe=false, 
            layoutvoffset=2em,
            bottom=2cm, top=3.5cm]{geometry}

\usepackage[spanish]{babel}
\usepackage[babel]{microtype}
\usepackage{lipsum}
\usepackage{unicode-math}
% Fonts can be customized here.
\defaultfontfeatures{Mapping=tex-text}
\setmainfont [Ligatures={Common}]{Linux Libertine O}
\setmonofont[Scale=0.9]{Linux Libertine Mono O}
%\usepackage[svgnames]{xcolor} % Gestión de colores
\usepackage{hyperref}
\hypersetup{
  colorlinks=true, linktocpage=true, pdfstartpage=3, pdfstartview=FitV,%
  breaklinks=true, pageanchor=true,%
  pdfpagemode=UseNone, %
  plainpages=false, bookmarksnumbered, bookmarksopen=true, bookmarksopenlevel=1,%
  hypertexnames=true, pdfhighlight=/O,%nesting=true,%frenchlinks,%
  urlcolor=Maroon, linkcolor=RoyalBlue, citecolor=Blue, %pagecolor=RoyalBlue,%
  pdftitle={},%
  pdfauthor={\textcopyright\ C. Manuel Carlevaro},%
  pdfsubject={},%
  pdfkeywords={},%
  pdfcreator={XeLaTeX},%
  pdfproducer={XeLaTeX}%
}

\usepackage{hyperref}
\usepackage{fontawesome, gensymb}
\usepackage{graphicx}
\setlength{\parindent}{3em}
\setlength{\parskip}{1em} 
\usepackage[shortlabels]{enumitem}

%\NewDocumentCommand{\evalat}{sO{\big}mm}{%
  %\IfBooleanTF{#1}
   %{\mleft. #3 \mright|_{#4}}
   %{#3#2|_{#4}}%
%}


\title{Cálculo avanzado}
\author{Dpto. de Ingenería Mecánica}
\date{Clase 5: Transformada de Laplace}


\begin{document}
% \maketitle

\begin{center}
\framebox[1.0\textwidth][c]{
\huge{\textsc{Cálculo Avanzado}} 
}
\end{center} 

\begin{center}
\vspace{\baselineskip}
\Large{\textsc{Departamento de Ingenería Mecánica}} \\
\textsc{Facultad Regional La Plata} \\
\textsc{Universidad Tecnológica Nacional}
\end{center}

% \vspace{1em}

\begin{center}
\begin{tabular}{r l}
    \textbf{Práctica:} & 5 \\
 \textbf{Tema:} & Transformadas de Laplace. Solución de problemas con valores iniciales.\\
 \textbf{Profesor Titular:} & Manuel Carlevaro \\
 \textbf{Jefe de Trabajos Prácticos:} & Diego Amiconi \\
 \textbf{Ayudante de Primera:} & Lucas Basiuk 
\end{tabular}\end{center}

\vspace{1em}

\begin{question} % Schaum's Outline ... Advanced Calculus 12.34b pg 338 
    Halle la transformada de Laplace de $f(t) = \cos t$.
\end{question}

\begin{question}
Demostrar que 
\[ \mathscr{L}(t^n) = \frac{n!}{s^{n + 1}}, \quad n = 0, 1, \cdots \] 

Sugerencia: a partir de $\mathscr{L}(1) = 1 / s, \, (n = 0)$, asumir por inducción que vale para cualquier entero $n \geq 0$ y mostrar que vale para $n + 1$.
\end{question}

\begin{question} % Kreyszig p. 217
La función escalón unitario o función de Heaviside se define como:
\[ H(t - a) = 
    \begin{cases}
    0 & \text{ si } t < a\\
    1 & \text{ si } t > a
    \end{cases}
\]
Halle la transformada de Laplace de $H(t - a)$.
\end{question}

\begin{question} % O'Neil Ejemplo 1.3 pg 11
    Resolver el siguiente problema con valores iniciales:
    \[ y' - 4 y = 1, \quad y(0) = 1 \]
\end{question}

\begin{question} % O'Neil Sección 1.2 problema 3 pg 14
    Resolver el siguiente problema con valores iniciales:
    \[ y' + 4 y = \cos t, \quad y(0) = 0 \]
\end{question}

\begin{question} % O'Neil Sección 1.2 problema 9 pg 14
    Resolver el siguiente problema con valores iniciales:
    \[ y'' + 16 y = 1 + t, \quad y(0) = -1, \quad y'(0) = 1 \]
\end{question}

\end{document}
