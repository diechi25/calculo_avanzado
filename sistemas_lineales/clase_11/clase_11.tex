\documentclass[9pt, aspectratio=169]{beamer}

\usetheme{metropolis}
\setbeamertemplate{itemize items}{\faAngleRight}

\metroset{titleformat=smallcaps,block=fill,numbering=counter,progressbar=frametitle,sectionpage=none}
\setbeamersize{text margin left=5mm,text margin right=5mm} 
% %%%%%%%%%%%%%%%%%%%%%%%%%%%%%%%%%%%%%%%%%%%%%%%%%%%%%%%%%%%%%%%%%%%%%%%%%%%%%%
% \embedvideo{<poster or text>}{<video file (MP4+H264)>}
% \embedvideo*{...}{...}                     % auto-play
%%%%%%%%%%%%%%%%%%%%%%%%%%%%%%%%%%%%%%%%%%%%%%%%%%%%%%%%%%%%%%%%%%%%%%%%%%%%%%

\usepackage[bigfiles]{pdfbase}
\ExplSyntaxOn
\NewDocumentCommand\embedvideo{smm}{
  \group_begin:
  \leavevmode
  \tl_if_exist:cTF{file_\file_mdfive_hash:n{#3}}{
    \tl_set_eq:Nc\video{file_\file_mdfive_hash:n{#3}}
  }{
    \IfFileExists{#3}{}{\GenericError{}{File~`#3'~not~found}{}{}}
    \pbs_pdfobj:nnn{}{fstream}{{}{#3}}
    \pbs_pdfobj:nnn{}{dict}{
      /Type/Filespec/F~(#3)/UF~(#3)
      /EF~<</F~\pbs_pdflastobj:>>
    }
    \tl_set:Nx\video{\pbs_pdflastobj:}
    \tl_gset_eq:cN{file_\file_mdfive_hash:n{#3}}\video
  }
  %
  \pbs_pdfobj:nnn{}{dict}{
    /Type/RichMediaInstance/Subtype/Video
    /Asset~\video
    /Params~<</FlashVars (
      source=#3&
      skin=SkinOverAllNoFullNoCaption.swf&
      skinAutoHide=true&
      skinBackgroundColor=0x5F5F5F&
      skinBackgroundAlpha=0
    )>>
  }
  %
  \pbs_pdfobj:nnn{}{dict}{
    /Type/RichMediaConfiguration/Subtype/Video
    /Instances~[\pbs_pdflastobj:]
  }
  %
  \pbs_pdfobj:nnn{}{dict}{
    /Type/RichMediaContent
    /Assets~<<
      /Names~[(#3)~\video]
    >>
    /Configurations~[\pbs_pdflastobj:]
  }
  \tl_set:Nx\rmcontent{\pbs_pdflastobj:}
  %
  \pbs_pdfobj:nnn{}{dict}{
    /Activation~<<
      /Condition/\IfBooleanTF{#1}{PV}{XA}
      /Presentation~<</Style/Embedded>>
    >>
    /Deactivation~<</Condition/PI>>
  }
  %
  \hbox_set:Nn\l_tmpa_box{#2}
  \tl_set:Nx\l_box_wd_tl{\dim_use:N\box_wd:N\l_tmpa_box}
  \tl_set:Nx\l_box_ht_tl{\dim_use:N\box_ht:N\l_tmpa_box}
  \tl_set:Nx\l_box_dp_tl{\dim_use:N\box_dp:N\l_tmpa_box}
  \pbs_pdfxform:nnnnn{1}{1}{}{}{\l_tmpa_box}
  %
  \pbs_pdfannot:nnnn{\l_box_wd_tl}{\l_box_ht_tl}{\l_box_dp_tl}{
    /Subtype/RichMedia
    /BS~<</W~0/S/S>>
    /Contents~(embedded~video~file:#3)
    /NM~(rma:#3)
    /AP~<</N~\pbs_pdflastxform:>>
    /RichMediaSettings~\pbs_pdflastobj:
    /RichMediaContent~\rmcontent
  }
  \phantom{#2}
  \group_end:
}
\ExplSyntaxOff
%%%%%%%%%%%%%%%%%%%%%%%%%%%%%%%%%%%%%%%%%%%%%%%%%%%%%%%%%%%%%%%%%%%%%%%%%%%%%%

\usepackage{fontspec,minted}
\usepackage[scale=1]{ccicons}
\usepackage{metalogo}
\usepackage{xcolor,colortbl}
\usepackage{multicol,multirow,booktabs}
\usepackage{appendixnumberbeamer}
\usepackage{graphicx}
\usepackage{bm}
\usepackage{fontawesome}
\usepackage{csquotes}
\usepackage[backend=biber, natbib, sorting=nyt, doi=true, url=false, url=false, isbn=false, maxbibnames=10]{biblatex}
\addbibresource{../../utils/refs.bib}

\usepackage[spanish]{babel}
\usepackage{mathtools}
\usefonttheme{professionalfonts}
\usepackage{textcomp}

\setsansfont[BoldFont={Iwona Bold}, Numbers={Lining, Proportional}]{Iwona Light}
% \setmathsfont(Digits)[Numbers={Lining, Proportional}]{Fira Sans Light}
\setmonofont[Scale=MatchLowercase]{DejaVu Sans Mono}

\setbeamercolor{alerted text}{fg=red,bg=black!2}
\setbeamercolor{progress bar}{fg=red,bg=red!2}
\setbeamertemplate{itemize item}{\faCaretRight}
\setbeamertemplate{itemize subitem}{ \faAngleRight}
\setbeamertemplate{blocks}[shadow=false]
\setbeamercolor{block title}{bg=black!30,fg=red}
\setbeamercolor{block body}{bg=black!20,fg=black}
 
\usepackage{gensymb,amssymb}
\usepackage{upquote}
\usepackage{algpseudocode}
\algrenewcommand\algorithmicrequire{\textbf{Requiere}}
\algrenewcommand\algorithmicensure{\textbf{Devuelve}}
%\setbeamertemplate{blocks}[rounded][shadow=false]
\setbeamertemplate{blocks}[shadow=false]

\newcommand{\cx}{\column{0.5\textwidth}}
\newcommand{\cw}[1]{\column{#1\textwidth}}

\author{Manuel Carlevaro}
\date{{\tiny Departamento de Ingeniería Mecánica \\[-1em]
             Grupo de Materiales Granulares - UTN FRLP \\
        \faEnvelope{} manuel.carlevaro@gmail.com \- $\cdot$ \- \faTwitter{} @mcarlevaro}}
\institute{
  \vspace{6em}
  \centering
  {\tiny
  Cálculo Avanzado \enspace • \enspace 2022 \\
    \faLinux \- $\cdot$ \- \fontspec{TeX Gyre Pagella}\XeLaTeX \- $\cdot$ \- \ccbysa }
}

%% Operadores
\DeclareMathOperator{\sen}{sen}
\DeclareMathOperator{\sign}{sign}
\newcommand{\T}[1]{\underline{\bm{#1}}}
\DeclareMathOperator{\Tr}{Tr}

\usepackage{hyperref}
\hypersetup{
    colorlinks,
    citecolor=blue,
    filecolor=black,
    linkcolor=blue,
    urlcolor=blue
}
\urlstyle{same}

%% Códigos
\usepackage{minted}
\newminted[cpp]{cpp}{linenos,fontsize=\footnotesize,frame=lines,numbersep=4pt}
\newmintedfile[cppcode]{cpp}{linenos,fontsize=\footnotesize,frame=lines,numbersep=4pt}
\newcommand{\mic}[1]{\mintinline{C++}{#1}}

\newminted[py]{python}{linenos,fontsize=\footnotesize,frame=lines,numbersep=4pt}
\newminted[pyc]{pycon}{linenos,fontsize=\footnotesize,frame=lines,numbersep=4pt} % Consola de Python
\newminted[ipy3]{ipython3}{linenos,fontsize=\footnotesize,frame=lines,numbersep=4pt} % Consola de iPython3
\newmintedfile[pycode]{python}{linenos,fontsize=\footnotesize,frame=lines,numbersep=4pt}

\newmintedfile[makef]{basemake}{linenos,fontsize=\footnotesize,frame=lines,numbersep=4pt}
\definecolor{bg}{RGB}{22,43,58}
\newminted[shell]{console}{linenos=false,fontsize=\footnotesize,breaklines=true, frame=single} % Linea de comandos
\renewcommand\listingscaption{Código}

\makeatletter
\AtBeginEnvironment{minted}{\dontdofcolorbox}
\def\dontdofcolorbox{\renewcommand\fcolorbox[4][]{##4}}
\makeatother

% uso:
% Ejemplo de uso explícito:
% \begin{py}
% >>> list("abcd")
% ['a', 'b', 'c', 'd']
% \end{py}
% 
% Ahora ejemplo de código en file:
% \pycode{Chapters/intro/code/hola.py}
% 
% También se puede poner un sector del file:
% \pycode[firstline=6, lastline=7]{Chapters/intro/code/hola.py}
% 
% También se puede poner código \textit{inline}: \mip{print('¡Hola mundo!')} y en una sola línea:
% \slp|if __name__ == '__main__')|
% 
% Por último, se puede poner el código en un entorno \textit{float}, esto es, como las tablas y las figuras, con un caption y un label para luego hacer referencias, como por ejemplo al Código \ref{code:hola}.


\usepackage{tikz}
\usetikzlibrary{shapes,shadows,arrows,positioning,matrix,chains,backgrounds,fit}

\tikzset{
    %Define standard arrow tip
    >=stealth',
    %Define style for boxes
    obj/.style={
           rectangle,
           rounded corners,
           draw, very thick,
           text width=10em, fill=green!20,
           minimum height=2em,
           text centered, drop shadow},
    proc/.style={
	    rectangle, rounded corners,
	    draw,fill=red!50,very thick,
	    text width=8em,minimum height=2em,
	    text centered, drop shadow},
    % Define arrow style
    pil/.style={
           ->,
           thick,
           shorten <=2pt,
           shorten >=2pt,}
}

\setbeamertemplate{bibliography item}{%
  \ifboolexpr{ test {\ifentrytype{book}} or test {\ifentrytype{mvbook}}
    or test {\ifentrytype{collection}} or test {\ifentrytype{mvcollection}}
    or test {\ifentrytype{reference}} or test {\ifentrytype{mvreference}} }
    {\setbeamertemplate{bibliography item}{\faBook}}
    {\ifentrytype{online}
            {\setbeamertemplate{bibliography item}{\faGlobe}}
   {\setbeamertemplate{bibliography item}{\faFileText}}}%
  \usebeamertemplate{bibliography item}}

\defbibenvironment{bibliography}
  {\list{}
     {\settowidth{\labelwidth}{\usebeamertemplate{bibliography item}}%
      \setlength{\leftmargin}{\labelwidth}%
      \setlength{\labelsep}{\biblabelsep}%
      \addtolength{\leftmargin}{\labelsep}%
      \setlength{\itemsep}{\bibitemsep}%
      \setlength{\parsep}{\bibparsep}}}
  {\endlist}
  {\item}
\newcommand{\bcite}[1]{\citeauthor{#1}, \citetitle{#1} (\citeyear{#1})}


\title{Solución de sistemas de ecuaciones lineales}
\subtitle{Condicionamiento. Métodos directos: eliimnación gaussiana, factorización LU. Pivoteo.}

%%%%
% Bibliografía
%%%%

\begin{document}
\maketitle

\begin{frame}
\begin{columns}[t]
    \cw{0.45}
\textbf{Resolver:}
\begin{align*}
    a_{11}x_1+a_{12}x_2+\cdots +a_{1n}x_n &= b_1 \\
    a_{21}x_2+a_{22}x_2+\cdots +a_{2n}x_n &= b_2 \\
                                   & \vdots \\
    a_{n1}x_1+a_{n2}x_2+\cdots +a_{nn}x_n &= b_n \\
 \end{align*} \pause
\vspace{-2em}

 O, en forma matricial:
 \begin{equation*}
  \mathbb{A} \mathbf{x} = \mathbf{b}
 \end{equation*}  \pause
\vspace{-1em}

  Matriz de coeficientes aumentada:
\begin{equation*}
 \left[ 
 \begin{array}{cccc|c}
 a_{11} & a_{12} & \ldots & a_{1n} & b_1 \\
 a_{21} & a_{22} & \ldots & a_{2n} & b_2\\
 \vdots & \vdots & \ddots & \vdots & \vdots\\
 a_{N1} & a_{N2} & \ldots & a_{nn} & b_n \\
 \end{array} \right] 
\end{equation*} \pause

\cw{0.45}

Si $\mathbb{A} \in \mathbb{R}^{n \mul n}$ y $\bm{b} \in \mathbb{R}$, la existencia y unicidad de la solución está asegurada si una de las siguientes condiciones se cumple:
\begin{itemize}
    \item $\mathbb{A}$ es invertible (no singular)
    \item $\rg(A) = n$
    \item El sistema homogéneo $\mathbb{A} \bm{x} = \bm{0}$ admite solo la solución nula.
\end{itemize}

\hrulefill \pause \vspace{1em}


\textbf{Solución:} regla de Cramer
\[ x_j = \frac{\Delta_j}{\det{\mathbb{A}}} \]
Esfuerzo computacional: $\bigO((n+ 1)!)$. \\
$n = 50$, Intel i7: 200 Gflops $\approx 5 \times 10^{45}$ años.

\end{columns}
\end{frame}

\begin{frame}
    \begin{columns}[t]
\cx
\textbf{Métodos:}
\begin{itemize}
    \item Directos: alcanzan la solución en un número finito de pasos. $\bigO(2/3N^3$).
        \begin{itemize}
            \item Eliminación gaussiana
            \item Factorización $LU$
            \item LDM$^T$
            \item Factorización de Cholesky
            \item Factorización $QR$
        \end{itemize}
    \item Iterativos: más eficientes en casos particulares. $\bigO(N^2$).
        \begin{itemize}
            \item Jacobi
            \item Gauss-Seidel
            \item Subespacios de Krylov
            \item GMRES
        \end{itemize}
\end{itemize}

\hrulefill \pause \vspace{1em}

\textbf{Estabilidad de la solución:} 
\begin{equation*}(\mathbb{A} + \delta \mathbb{A}) (\bm{x} + \textcolor{red}{\delta \bm{x}}) = \bm{b} + \delta \bm{b} \end{equation*}

Unicidad de la solución: $\mathbb{A}$ es \alert{no singular}: $|\mathbb{A}| \neq 0$.

Número de condición: $\cond(\mathbb{A}) = \norm{\mathbb{A}} \norm{\mathbb{A}^{-1}}$.

\cx

%¿Que ocurre si $\mathbb{A}$ es \textit{casi} singular ($|\mathbb{A}|$ pequeño)? El determinante es pequeño si $|\mathbb{A}| \ll \norm{\mathbb{A}}$. \pause


Se puede demostrar\footnotemark[1], que si 

\[ \cond(\mathbb{A}) \frac{\norm{\delta \mathbb{A}}}{\norm{\mathbb{A}}} < 1 \]
se cumple:

\[ \frac{\norm{\delta \bm{x}}}{\norm{\bm{x}}} \leq \frac{\cond(\mathbb{A})}{1 - \cond(\mathbb{A}) \frac{\norm{\delta \mathbb{A}}}{\norm{\mathbb{A}}}} \left( \frac{\norm{\delta \bm{b}}}{\norm{\bm{b}}} + \frac{\norm{\delta \mathbb{A}}}{\norm{\mathbb{A}}} \right) \]

En general es muy costoso evaluar $\norm{\mathbb{A}}$. Usualmente se compara $|\mathbb{A}|$ con $a_{ij}$. 


\textbf{Ejemplo:}
\[
\begin{system}
2 x + y = 3 \\ 2 x + 1.001 y = 0
\end{system}, \quad |\mathbb{A}| = 0.002 \]
Solución: $x = 1501.5, y = -3000$. 

\footnotetext[1]{Ver \bcite{moreno2014}, sección 2.3., y \bcite{quarteroni2000}, sección 3.1.}
\end{columns}
\end{frame}

\begin{frame}[fragile]
\begin{columns}[t]
\cx
\[ \begin{system}
2 x + y = 3 \\ 2x + 1.002 y = 0
\end{system}, \quad |\mathbb{A}| = 0.004 \] 
Solución: $x = 751.5, y = -1500$.

\fbox{0.1\% de cambio en $a_{ij} \mapsto$ 100\% de cambio en $\bm{x}$.}\\
\begin{alertblock}{Mal condicionamiento}
    Si la solución de un sistema lineal cambia mucho cuando el problema cambia muy poco, la matriz está \alert{mal condicionada}.
\end{alertblock} \pause

\textbf{Otro ejemplo:}
\[ \mathbb{A} = \begin{bmatrix} 1 & 100 \\ 0 & 1 \end{bmatrix}, \quad |\mathbb{A}| = 1 \]
\[ \mathbb{A}^{-1} = \begin{bmatrix} 1 & -100 \\ 0 & 1 \end{bmatrix}, \quad |\mathbb{A}^{-1}| = 1 \]
\pause
\cx
\textbf{Soluciones:}
\[ \bm{b} = \begin{bmatrix} 100 \\ 1 \end{bmatrix} \rightarrow \bm{x} = \begin{bmatrix} 0 \\ 1 \end{bmatrix}, \; \bm{b} = \begin{bmatrix} 100 \\ 0 \end{bmatrix} \rightarrow \bm{x} = \begin{bmatrix} 100 \\ 0 \end{bmatrix}  \]
\fbox{1\% de cambio en $\bm{b} \mapsto$ 100\% de cambio en $\bm{x}$.} \pause

\pycode{code/cond.py}
\begin{shell}
$ ./cond.py 
||a|| = 100.00999900019995, ||ai|| = 100.00999900019995
cond(a) = 10001.999900019995
cond(a) = 10001.999900019995
\end{shell}
\end{columns}
\end{frame}

\begin{frame}
\begin{columns}[c]
\cw{0.3}
\textbf{Métodos directos:}
\begin{itemize}
    \item $a_{ik} \rightleftarrows a_{jk}, |\mathbb{A}| \rightarrow -|\mathbb{A}|$
    \item $k \mul a_{ij}, |\mathbb{A}| \rightarrow k \mul |\mathbb{A}|$
    \item $a_{ik} - k \, a_{ik}, |\mathbb{A}|$ no cambia
\end{itemize} \pause
\cw{0.6}
 \begin{center}
\begin{tabular}{c c c}
\toprule
\textbf{Método} & \textbf{Forma inicial} & \textbf{Forma final} \\
\midrule
Eliminación gaussiana & $\mathbb{A} \mathbf{x} = \mathbf{b}$ & $\mathbb{U} \mathbf{x} = \mathbf{c}$ \\
Descomposición LU & $\mathbb{A} \mathbf{x} = \mathbf{b}$ & $ \mathbb{LU} \mathbf{x} = \mathbf{b}$ \\
Eliminación de Gauss-Jordan & $\mathbb{A} \mathbf{x} = \mathbf{b}$ & $\mathbb{I} \mathbf{x} = \mathbf{c}$ \\
\bottomrule
\end{tabular} 
\end{center}
\end{columns} \pause

\vspace{2em}

\begin{columns}[t]
\cw{0.4}
 Matriz triangular superior 3x3
\[
    \mathbb{U} = \begin{bmatrix} 
 U_{11} & U_{12} &  U_{13} \\
 0      & U_{22} &  U_{23} \\
 0      & 0      &  U_{33}
 \end{bmatrix} 
\]

\cw{0.4}
Matriz triangular inferior 3x3
\[
    \mathbb{L} = \begin{bmatrix} 
 L_{11} & 0      & 0 \\
 L_{21} & L_{22} & 0 \\
 L_{31} & L_{32} & L_{33}
 \end{bmatrix} \] 
\end{columns}
\end{frame}

\begin{frame}
\begin{columns}[t]
\cx
\textbf{Eliminación gaussiana}

Dos fases: eliminación y solución.

\textbf{Fase de eliminación:}

Utiliza solo una operación elemental:
\[ \text{Ec}(i) \leftarrow \text{Ec}(i) - \lambda \mul \text{Ec}(j) \]
donde Ec($j$) se denomina \textit{ecuación pivote}. \pause
\vspace{1em}

\textbf{Ejemplo:}
\[ \begin{system}
 4x_1-2x_2+x_3 = 11 \\
 -2x_1+4x_2-2x_3 = -16 \\
  x_1-2x_2+4x_3 = 17 \\
\end{system} \]
\[ \left[ \begin{matrix} 4 & -2 & 1 \\ -2 & 4 & -2 \\ 1 & -2 & 4 \end{matrix}
 \; \middle| \;
\begin{matrix} 11 \\ -16 \\ 17 \end{matrix} \right] \]
\pause

\cx
Ec(1): pivote; \\
Ec(2) $\leftarrow$ Ec(2) $-(-0.5) \times$ Ec(1);\\
Ec(3) $\leftarrow$ Ec(3) $-0.25 \times$ Ec(1)

\[ \left[ \begin{matrix} 4 & -2 & 1 \\ 0 & 3 & -1.5 \\ 0 & -1.5 & 3.75 \end{matrix}
 \; \middle| \;
\begin{matrix} 11 \\ -10.5 \\ 14.25 \end{matrix} \right] \] \pause

Ec(2): pivote; \\
Ec(3) $\leftarrow$ Ec(3) $-(-0.5) \times$ Ec(2)
\[ \left[ \begin{matrix} 4 & -2 & 1 \\ 0 & 3 & -1.5 \\ 0 & 0 & 3 \end{matrix}
 \; \middle| \;
\begin{matrix} 11 \\ -10.5 \\ 9 \end{matrix} \right] \] \pause

\alert{Bonus:} no se altera $|\mathbb{A}|$:
\[ |\mathbb{A}| = |\mathbb{U}| = U_{11} \times U_{22} \times \cdots U_{NN} \]

\textbf{Fase de solución:} sustitución hacia atrás.
\end{columns}
\end{frame}

\begin{frame}
\textbf{Algoritmo:}
\begin{equation*} {\small
\left[ 
 \begin{array}{ccccccccc|c}
 a_{11} & a_{12} & a_{13} & \cdots & a_{1k} & \cdots & a_{1j} & \cdots & a_{1n} & b_1 \\
 0      & a_{22} & a_{23} & \cdots & a_{2k} & \cdots & a_{2j} & \cdots & a_{2n} & b_2 \\
 \vdots & \vdots & \vdots & \ddots & \vdots & \ddots & \vdots & \ddots & \vdots & \vdots \\
 0      & 0      & 0      & \cdots & a_{kk} & \cdots & a_{kj} & \cdots & a_{kn} & b_k \\
 \hline
\vdots & \vdots & \vdots & \ddots & \vdots & \ddots & \vdots & \ddots & \vdots & \vdots \\
 0      & 0      & 0      & \cdots & a_{ik} & \cdots & a_{ij} & \cdots & a_{in} & b_i \\
\vdots & \vdots & \vdots & \ddots & \vdots & \ddots & \vdots & \ddots & \vdots & \vdots \\
 0      & 0      & 0      & \cdots & a_{nk} & \cdots & a_{nj} & \cdots & a_{nn} & b_N \\
 \end{array} \right] }
\end{equation*}
\begin{columns}[t]
\column{5cm}
\textbf{Eliminación:} {\small
\begin{algorithmic}[1]
 \For{$i \gets k+1,k+2, \ldots n$}
    \State $\lambda \gets a_{ik}/a_{kk}$
    \For{$j \gets k,n$}
      \State $a_{ij} \gets a_{ij} - \lambda a_{kj}$
    \EndFor
    \State $b_i \gets b_i - \lambda b_k$
 \EndFor	
\end{algorithmic} }
\column{6cm}
\textbf{Solución:} {\small
\begin{eqnarray*}
 x_n &=& b_n / a_{nn} \\
 x_k &=& \left( b_k- \sum_{j=k+1}^n a_{kj}x_j \right)  \dfrac{1}{a_{kk}} \\
 & &k = n-1, n-2, \ldots, 1
\end{eqnarray*} }
\end{columns}
\end{frame}


\begin{frame}
\textbf{Factorización o descomposición LU:}
 \begin{equation*}
  \mathbb{A} = \mathbb{L U} 
 \end{equation*}
 
 \begin{center}
\begin{tabular}{c c}
\toprule
\textbf{Nombre} & \textbf{Restricción} \\
\midrule
Descomposición de Doolittle & $L_{ii} = 1, \; i = 1, 2, \cdots,n$ \\
Descomposición de Crout & $U_{ii} = 1, \; i = 1, 2, \cdots,n$ \\
Descomposición de Choleski & \multirow{2}{*}{$\mathbb{L} = \mathbb{U}^T$} \\
{\small ($\mathbb{A}$ debe ser simétrica y definida positiva)} & \\
\bottomrule
\end{tabular} 
\end{center}
\vspace{2em}

\begin{columns}[t]
\cw{0.4}
Luego de la factorización:
\begin{equation*}
 \mathbb{LU} \, \bm{x} = \bm{b}
\end{equation*}
La solución consiste en:
\begin{equation*}
 \mathbb{L} \, \bm{y} = \bm{b}
\end{equation*}

\cw{0.4}
resuelta por sustitución hacia adelante, seguida de:
\begin{equation*}
 \mathbb{U} \, \bm{x} = \bm{y}
\end{equation*}
que da el resultado $\bm{x}$ obtenido por sustitución hacia atrás.
\end{columns}
\end{frame}

\begin{frame}
\textbf{Método de Doolittle:}
 \begin{equation*}
\mathbb{L} = \left[ 
 \begin{array}{ccc}
 1 & 0 & 0 \\
 L_{21} & 1 & 0 \\
 L_{31} & L_{32} &1 \\
 \end{array} \right] \hspace{1cm} 
\mathbb{U} = \left[ 
 \begin{array}{ccc}
 U_{11} & U_{12} & U_{13} \\
 0 & U_{22} & U_{23} \\
 0 & 0 & U_{33} \\
 \end{array} \right]
 \end{equation*}
Encontramos $\mathbb{A}$:
\begin{equation*}
 \mathbb{A}= \left[ \begin{array}{lll}
 U_{11} & U_{12} & U_{13} \\
 U_{11} L_{21} & U_{12} L_{21}+U_{22} & U_{13} L_{21} + U_{23} \\
 U_{11} L_{31} & U_{12} L_{31}+U_{22}L_{32} & U_{13} L_{31}+U_{23} L_{32} + U_{33} \\
 \end{array} \right]
\end{equation*} \pause
Aplicamos ahora eliminación gaussiana con las siguientes operaciones elementales:
\begin{center}
 fila 2 $\leftarrow$ fila 2 $- L_{21} \times$ fila 1 (elimina $a_{21}$) \\
 fila 3 $\leftarrow$ fila 3 $- L_{31} \times$ fila 1 (elimina $a_{31}$)
\end{center}
\begin{equation*}
 \mathbb{A}^{\prime}= \left[ \begin{array}{ccc}
 U_{11} & U_{12} & U_{13} \\
 0 & U_{22} & U_{23} \\
 0 & U_{22}L_{32} & U_{23} L_{32} + U_{33} \\
 \end{array} \right]
\end{equation*}
\end{frame}

\begin{frame}
 \textbf{Método de Doolittle (cont.):} tomamos ahora la segunda fila como pivote:
\begin{columns}
\cx
\begin{equation*}
 \mathbb{A}^{\prime}= \left[ \begin{array}{ccc}
 U_{11} & U_{12} & U_{13} \\
 0 & U_{22} & U_{23} \\
 0 & U_{22}L_{32} & U_{23} L_{32} + U_{33} \\
 \end{array} \right]
\end{equation*}

\cx
 \begin{center}
  fila 3 $\leftarrow$ fila 3 $- L_{32} \times$ fila 2 (elimina $a_{32}$)
 \end{center}
 \begin{equation*}
\mathbb{A}^{\prime \prime} = \mathbb{U} = \left[ 
 \begin{array}{ccc}
 U_{11} & U_{12} & U_{13} \\
 0 & U_{22} & U_{23} \\
 0 & 0 & U_{33} \\
 \end{array} \right]
  \end{equation*}
\end{columns}
 \bigskip \pause
 \begin{columns}[t]
\cw{0.4}
\textbf{Características del método de Doolittle:} \medskip
 
 \begin{itemize} {\small
  \item La matriz $\mathbb{U}$ es idéntica a la matriz triangular superior que resulta de la eliminación gaussiana
  \medskip
  
  \item Los elementos no diagonales de $\mathbb{L}$ son los factores que multiplican a la ecuación pivote durante la eliminación gaussiana: $L_{ij}$ es el multiplicador que elimina $a_{ij}$ }
 \end{itemize} \pause
\cw{0.4}
\textbf{Almacenamiento:} \bigskip
 \begin{equation*}
\mathbb{L}/\mathbb{U} = \left[ 
 \begin{array}{ccc}
 U_{11} & U_{12} & U_{13} \\
 L_{21} & U_{22} & U_{23} \\
 L_{31} & L_{32} & U_{33} \\
 \end{array} \right]
  \end{equation*}
 \end{columns}
\end{frame}

\section{Pivoteo}
%
%   Pivoteo
%   http://math.fullerton.edu/mathews/n2003/PivotingMod.html
%   (no funciona 2019.10.04)
%
%   VER presentación sobre pivoteo en /docs
%
%   Numerical Methods in Engineering with Python 
%   J. Kiusalaas
%   pg 75 del PDF
%
\begin{frame}[containsverbatim, fragile]{Pivoteo}
Sistema:
\begin{align*}
 2 x_1 - \phantom{2} x_2 \phantom{ + x_3}\;\; &= 1 \\
 -x_1 + 2 x_2 - x_3 &= 0 \\
\phantom{x_1} -x_2 + x_3 &= 0 \\
\end{align*}
Solución: \( x_1 = x_2 = x_3 = 1 \) \bigskip
\begin{columns}[c]
 \column{5cm}
  \begin{equation*}
    \left[ 
    \begin{array}{ccc|c}
    2 & -1 & 0 & 1 \\
    -1 & 2 & -1 & 0 \\
    0 & -1 & 1 & 0\\
    \end{array} \right] 
  \end{equation*}
  \begin{center} OK \end{center} \pause
 \column{5cm}
  \begin{equation*}
    \left[ 
    \begin{array}{ccc|c}
    0 & -1 & 1 & 0\\
    -1 & 2 & -1 & 0 \\
    2 & -1 & 0 & 1 \\
    \end{array} \right] 
  \end{equation*}
  \begin{center} \alert{NOK} \end{center} 
\end{columns}
\end{frame}

\begin{frame}[containsverbatim, fragile]{Pivoteo}
\begin{equation*} [\mathbb{A} | \mathbf{b}] = 
  \left[
  \begin{array}{ccc|c}
    \varepsilon & -1 & 1 & 0 \\
    -1 & 2 & -1 & 0 \\
    2 & -1 & 0 & 1\\
  \end{array} \right]
 \rightarrow [\mathbb{A'} | \mathbf{b'}] = 
    \left[ 
    \begin{array}{ccc|c}
    \varepsilon & -1 & 1 & 0 \\
    0 & 2-1/\varepsilon & -1+1/\varepsilon & 0 \\
    0 & -1+2/\varepsilon & -2/\varepsilon  & 1\\
    \end{array} \right] 
\end{equation*} \pause

Almacenamiento en la memoria ($\varepsilon \ll 1$):
\begin{equation*}
 [\mathbb{A'} | \mathbf{b'}] = 
    \left[ 
    \begin{array}{ccc|c}
    \varepsilon & -1 & 1 & 0 \\
    0 & -1/\varepsilon & 1/\varepsilon & 0 \\
    0 & 2/\varepsilon & -2/\varepsilon  & 1\\
    \end{array} \right] 
\end{equation*} \pause
$\mathbb{A}$ es \textit{diagonalmente dominante} si:
\[ |a_{ii}| > \sum_{\substack{j=1 \\j \neq i}}^N |a_{ij}|; \> (i = 1, 2, \ldots, N)  \]
\vspace{-0.1cm}
\begin{columns}
 \column{5cm}
 \begin{equation*}
  \left[ \begin{array}{ccc}
          -2 & 4 & -1 \\
          1 & -1 & 3 \\
          4 & -2 & 1
         \end{array} \right]
 \end{equation*}
 \begin{center}
  {\small NO diagonalmente dominante}
 \end{center}
 \column{5cm}
 \begin{equation*}
  \left[ \begin{array}{ccc}
          4 & -2 & 1 \\
          -2 & 4 & -1 \\
          1 & -1 & 3 
         \end{array} \right]
 \end{equation*}
 \begin{center}
  {\small Diagonalmente dominante}
 \end{center}
 \end{columns}
\end{frame}

\begin{frame}[containsverbatim,fragile]{Pivoteo: estrategias}
En todos los casos: si $a_{ii} \ne 0 \mapsto$ no intercambiar filas. \medskip

 \begin{itemize}
  \item \textbf{Pivoteo trivial}: 
  \begin{itemize}
   \item $a_{ii} = 0 \mapsto$ buscar el primer $a_{ki} \ne 0 (k > i)$ e intercambiar filas $i \leftrightarrows k$.
  \end{itemize} \medskip \pause

  \item \textbf{Pivoteo parcial}: 
  \begin{itemize}
   \item $a_{ii} = 0 \mapsto$ buscar la fila $k$ tal que $|a_{ki}| = \max \limits_{j > i} |a_{ji}| \wedge a_{ji} \ne 0$ e intercambiar filas $i  \leftrightarrows k$.
  \end{itemize}  \medskip \pause

  \item \textbf{Pivoteo parcial escalado}:
  \begin{itemize}
   \item Calcular $s_i = \max \limits_{1 \leq j \leq N} |a_{ij}|, \; i=1,\ldots,N$
   \item $a_{ii} = 0 \mapsto$ buscar la fila $k$ tal que $\dfrac{|a_{ki}|}{s_k} = \max \limits_{j > i} \dfrac{|a_{ji}|}{s_j} \wedge a_{ji} \ne 0$ e intercambiar filas $i \leftrightarrows k$ y $s_i \leftrightarrows s_k$.
  \end{itemize} \medskip \pause

  \item \textbf{Pivoteo completo o maximal}:
  \begin{itemize}
   \item $a_{ii} = 0 \mapsto$ buscar la fila $j>i$ y columna $k>i$ tal que $|a_{jk}| = \max \limits_{\substack{l > i \\ m > i}} |a_{lm}| \wedge a_{jk} \ne 0$ e intercambiar filas $i \leftrightarrows j$ y columnas $i \leftrightarrows k$.
  \end{itemize}
 \end{itemize} \pause
\begin{center}
\framebox[1.1\width]{\alert{Nota:} en matemática ``$x=0,\, x\neq 0$'', en \textit{mundo real} ``$|x|<\varepsilon, \, |x| > \varepsilon$''.}                                                                        \end{center}
\end{frame}

\begin{frame}[fragile]
\begin{columns}[c]
\cx
\textbf{ Ejemplo con Python: }
\pycode{code/ejemplo.py}

\cx
\begin{shell}
$ ./ejemplo.py 
x =  [ 2. -2.  9.]
a @ x ? [ True  True  True]
a @ x ? True
a @ x = [ 2.  4. -1.]
[[1. 0. 0.]
 [0. 0. 1.]
 [0. 1. 0.]]
[[ 1.          0.          0.        ]
 [ 0.          1.          0.        ]
 [ 0.33333333 -0.33333333  1.        ]]
[[3.         2.         0.        ]
 [0.         5.         1.        ]
 [0.         0.         0.33333333]]
\end{shell}
\end{columns}
\end{frame}


\section*{Bibliografía}
\begin{frame}[allowframebreaks]{Lecturas recomendadas}
\begin{itemize}
    \item \fullcite{burden2017}. Capítulo 6.
    \item \fullcite{strang2006}. Capítulo 7.
    \item \fullcite{salgado2023}. Capítulo 3.
    \item \fullcite{quarteroni2000}. Capítulo 3.
\end{itemize}
\end{frame}

\end{document}

