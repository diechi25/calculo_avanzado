\documentclass[11pt]{article}

\usepackage{exsheets}
\usepackage[paper=a4paper, headheight=110pt,showframe=false, 
            layoutvoffset=2em,
            bottom=2cm, top=3.5cm]{geometry}

\usepackage[spanish]{babel}
\usepackage[babel]{microtype}
\usepackage{lipsum}
\usepackage{unicode-math}
% Fonts can be customized here.
\defaultfontfeatures{Mapping=tex-text}
\setmainfont [Ligatures={Common}]{Linux Libertine O}
\setmonofont[Scale=0.9]{Linux Libertine Mono O}
%\usepackage[svgnames]{xcolor} % Gestión de colores
\usepackage{hyperref}
\hypersetup{
  colorlinks=true, linktocpage=true, pdfstartpage=3, pdfstartview=FitV,%
  breaklinks=true, pageanchor=true,%
  pdfpagemode=UseNone, %
  plainpages=false, bookmarksnumbered, bookmarksopen=true, bookmarksopenlevel=1,%
  hypertexnames=true, pdfhighlight=/O,%nesting=true,%frenchlinks,%
  urlcolor=Maroon, linkcolor=RoyalBlue, citecolor=Blue, %pagecolor=RoyalBlue,%
  pdftitle={},%
  pdfauthor={\textcopyright\ C. Manuel Carlevaro},%
  pdfsubject={},%
  pdfkeywords={},%
  pdfcreator={XeLaTeX},%
  pdfproducer={XeLaTeX}%
}

\usepackage{hyperref}
\usepackage{fontawesome, gensymb}
\usepackage{graphicx}
\setlength{\parindent}{3em}
\setlength{\parskip}{1em} 
\usepackage[shortlabels]{enumitem}

%\NewDocumentCommand{\evalat}{sO{\big}mm}{%
  %\IfBooleanTF{#1}
   %{\mleft. #3 \mright|_{#4}}
   %{#3#2|_{#4}}%
%}


\title{Cálculo avanzado}
\author{Dpto. de Ingenería Mecánica}
\date{Clase 14: Aproximación continua por mínimos cuadrados}


\begin{document}
% \maketitle

\begin{center}
\framebox[1.0\textwidth][c]{
\huge{\textsc{Cálculo Avanzado}} 
}
\end{center} 

\begin{center}
\vspace{\baselineskip}
\Large{\textsc{Departamento de Ingenería Mecánica}} \\
\textsc{Facultad Regional La Plata} \\
\textsc{Universidad Tecnológica Nacional}
\end{center}

% \vspace{1em}

\begin{center}
\begin{tabular}{r l}
    \textbf{Práctica:} & 14 \\
 \textbf{Tema:} & Aproximación continua por mínimos cuadrados. \\
 \textbf{Profesor Titular:} & Manuel Carlevaro \\
 \textbf{Jefe de Trabajos Prácticos:} & Diego Amiconi \\
 \textbf{Ayudante de Primera:} & Lucas Basiuk 
\end{tabular}\end{center}

\vspace{1em}

\begin{question} % Burden ej 8.2 1) a, b, d y e pg 185
Encuentre la aproximación lineal por mínimos cuadrados para $f(x)$ en el intervalo indicado si:
\begin{enumerate}[a)]
    \item $f(x) = x^2 + 3 x + 2$, en $[0, 1]$
    \item $f(x) = x^3$ en $[0, 2]$
    \item $f(x) = e^x$, en $[0, 2]$
    \item $f(x) = \frac{1}{2} \cos x + \frac{1}{3} \sen x$, en $[0, 1]$
\end{enumerate}
\end{question}

\begin{question} % Burden ej 8.2 3) pg 186
    Encuentre la aproximación polinomial por mínimos cuadrados de grado 2 para las funciones e intervalos del Ejercicio 1.
\end{question}

\begin{question} % Burden ej 8.2 2) pg 186
    Encuentre la aproximación polinomial por mínimos cuadrados de grado 2 para las funciones del Ejercicio 1 en el intervalo $[-1, 1]$.

\begin{question} % Epperson Example 4.14 pg 244 
    Construya una aproximación de mínimos cuadrados de cuarto grado a la función exponencial
    \[ f(x) = e^x \]
    sobre el intervalo $[-1, 1]$ usando polinomios de Legedre.
\end{question}

\begin{question} % Moreno Gonzalez. Ej. 40 pg. 119
Determinar la mejor aproximación a $x^3$ con un polinomio de segundo grado usando polinomios de Chebishev.
\end{question}


\begin{question} % Moreno Gonzalez. Ej. 42 pg. 120
Hallar la recta que mejor aproxima la gráfica de la función 
\[ f(x) = \frac{1}{1 + x^2} \]
con la norma inducida por el producto interno:
\[ \langle f, g \rangle = \int_0^5 f(x) g(x) \, dx \] 
usando una base de polinomios ortogonales.
\end{question}

\end{document}
