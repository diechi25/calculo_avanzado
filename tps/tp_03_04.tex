\documentclass[11pt]{article}

\usepackage{exsheets}
\usepackage[paper=a4paper, headheight=110pt,showframe=false, 
            layoutvoffset=2em,
            bottom=2cm, top=3.5cm]{geometry}

\usepackage[spanish]{babel}
\usepackage[babel]{microtype}
\usepackage{lipsum}
\usepackage{unicode-math}
% Fonts can be customized here.
\defaultfontfeatures{Mapping=tex-text}
\setmainfont [Ligatures={Common}]{Linux Libertine O}
\setmonofont[Scale=0.9]{Linux Libertine Mono O}
%\usepackage[svgnames]{xcolor} % Gestión de colores
\usepackage{hyperref}
\hypersetup{
  colorlinks=true, linktocpage=true, pdfstartpage=3, pdfstartview=FitV,%
  breaklinks=true, pageanchor=true,%
  pdfpagemode=UseNone, %
  plainpages=false, bookmarksnumbered, bookmarksopen=true, bookmarksopenlevel=1,%
  hypertexnames=true, pdfhighlight=/O,%nesting=true,%frenchlinks,%
  urlcolor=Maroon, linkcolor=RoyalBlue, citecolor=Blue, %pagecolor=RoyalBlue,%
  pdftitle={},%
  pdfauthor={\textcopyright\ C. Manuel Carlevaro},%
  pdfsubject={},%
  pdfkeywords={},%
  pdfcreator={XeLaTeX},%
  pdfproducer={XeLaTeX}%
}

\usepackage{hyperref}
\usepackage{fontawesome, gensymb}
\usepackage{graphicx}
\setlength{\parindent}{3em}
\setlength{\parskip}{1em} 
\usepackage[shortlabels]{enumitem}

%\NewDocumentCommand{\evalat}{sO{\big}mm}{%
  %\IfBooleanTF{#1}
   %{\mleft. #3 \mright|_{#4}}
   %{#3#2|_{#4}}%
%}


\title{Cálculo avanzado}
\author{Dpto. de Ingenería Mecánica}
\date{Trabajos prácticos 01 y 02}


\begin{document}
% \maketitle

\begin{center}
\framebox[1.0\textwidth][c]{
\huge{\textsc{Cálculo Avanzado}} 
}
\end{center} 

\begin{center}
\vspace{\baselineskip}
\Large{\textsc{Departamento de Ingenería Mecánica}} \\
\textsc{Facultad Regional La Plata} \\
\textsc{Universidad Tecnológica Nacional}
\end{center}

% \vspace{1em}

\begin{center}
\begin{tabular}{r l}
   \textbf{Trabajos prácticos} & 3 y 4 \\
 \textbf{Temas:} & Cálculo de raíces. Autovalores y autovectores\\
 \textbf{Profesor Titular:} & Manuel Carlevaro \\
 \textbf{Jefe de Trabajos Prácticos:} & Diego Amiconi \\
 \textbf{Ayudante de Primera:} & Lucas Basiuk 
\end{tabular}\end{center}

\vspace{1em}

\section{Trabajo práctico 03: Raíces de ecuaciones.}

\subsection{Método de bisección}
% Bradie Instructor manual Ch2s1 - p16a, pg 11.
Utilice el método de bisección para aproximar todos los ceros reales de la función
\[ f(x) = e^x + x^2 - x -4 \]
Utilice una tolerancia absoluta de $10^{-5}$ como criterio de finalización. Muestre en forma de tabla los valores intermedios calculados. Estime la cantidad de pasos necesarios para obtener una precisión de $10^{-8}$.

\subsection{Método de Newton-Raphson}
Aproxime los ceros de la función del problema anterior utilizando el método de Newton-Raphson. Compare la velocidad de convergencia de ambos métodos al iniciarlos con la estimación inicial en algunos de los extremos de los intervalos utilizados con el método de bisección.

\section{Trabajo práctico 04: Autovalores y autovectores.}

\subsection{Autovalores y autovectores} % Shaum's Outline of Linear Algebra, p 9.16 pg 315
Considere la matriz 
\[
    \bm{A} = \begin{bmatrix}
        3 & -1 & 1 \\
        7 & -5 & 1 \\
    6 & -6 & 2 \end{bmatrix} \]
    \begin{enumerate}[a)]
        \item Encuentre los autovalores de $\bm{A}$.
        \item Determine los autovectores asociados.
    \end{enumerate}

\subsection{Método de las potencias}
Utilice el método de las potencias para hallar el autovalor y el autovector dominantes del problema 2.1.

\subsection{Método QR}
Utilice el método QR para obtener los autovalores de la matriz $\bm{A}$ del problema 2.1.

\end{document}
